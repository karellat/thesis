%%% Hlavní soubor. Zde se definují základní parametry a odkazuje se na ostatní části. %%%

%% Verze pro jednostranný tisk:
% Okraje: levý 40mm, pravý 25mm, horní a dolní 25mm
% (ale pozor, LaTeX si sám přidává 1in)
\documentclass[12pt,a4paper]{report}
\setlength\textwidth{145mm}
\setlength\textheight{247mm}
\setlength\oddsidemargin{15mm}
\setlength\evensidemargin{15mm}
\setlength\topmargin{0mm}
\setlength\headsep{0mm}
\setlength\headheight{0mm}
% \openright zařídí, aby následující text začínal na pravé straně knihy
\let\openright=\clearpage

%% Pokud tiskneme oboustranně:
% \documentclass[12pt,a4paper,twoside,openright]{report}
% \setlength\textwidth{145mm}
% \setlength\textheight{247mm}
% \setlength\oddsidemargin{14.2mm}
% \setlength\evensidemargin{0mm}
% \setlength\topmargin{0mm}
% \setlength\headsep{0mm}
% \setlength\headheight{0mm}
% \let\openright=\cleardoublepage

%% Vytváříme PDF/A-2u
\usepackage[a-2u]{pdfx}

%% Přepneme na českou sazbu a fonty Latin Modern
\usepackage[czech]{babel}
\usepackage{lmodern}
\usepackage[T1]{fontenc}
\usepackage{textcomp}

%% Použité kódování znaků: obvykle latin2, cp1250 nebo utf8:
\usepackage[utf8]{inputenc}

%%% Další užitečné balíčky (jsou součástí běžných distribucí LaTeXu)
\usepackage{amsmath}        % rozšíření pro sazbu matematiky
\usepackage{amsfonts}       % matematické fonty
\usepackage{amsthm}         % sazba vět, definic apod.
\usepackage{bbding}         % balíček s nejrůznějšími symboly
			    % (čtverečky, hvězdičky, tužtičky, nůžtičky, ...)
\usepackage{bm}             % tučné symboly (příkaz \bm)
\usepackage{graphicx}       % vkládání obrázků
\usepackage{fancyvrb}       % vylepšené prostředí pro strojové písmo
\usepackage{indentfirst}    % zavede odsazení 1. odstavce kapitoly
\usepackage{natbib}         % zajištuje možnost odkazovat na literaturu
			    % stylem AUTOR (ROK), resp. AUTOR [ČÍSLO]
\usepackage[nottoc]{tocbibind} % zajistí přidání seznamu literatury,
                            % obrázků a tabulek do obsahu
\usepackage{icomma}         % inteligetní čárka v matematickém módu
\usepackage{dcolumn}        % lepší zarovnání sloupců v tabulkách
\usepackage{booktabs}       % lepší vodorovné linky v tabulkách
\usepackage{paralist}       % lepší enumerate a itemize
\usepackage[usenames]{xcolor}  % barevná sazba
\usepackage{listings}

%% Moje package
\usepackage{lipsum}                     % Dummytext
\usepackage{xargs}                      % Use more than one optional parameter in a new commands
% 
\usepackage[colorinlistoftodos,prependcaption,textsize=tiny]{todonotes}

%% Moje package
\usepackage{subcaption}

%%% Údaje o práci

% Název práce v jazyce práce (přesně podle zadání)
\def\NazevPrace{Evoluční algoritmy pro řízení heterogenních robotických swarmů}

% Název práce v angličtině
\def\NazevPraceEN{Evolutionary Algorithms for the Control of Heterogeneous Robotic Swarms}

% Jméno autora
\def\AutorPrace{Tomáš Karella}

% Rok odevzdání
\def\RokOdevzdani{2018}

% Název katedry nebo ústavu, kde byla práce oficiálně zadána
% (dle Organizační struktury MFF UK, případně plný název pracoviště mimo MFF)
\def\Katedra{Katedra teoretické informatiky a matematické logiky}
\def\KatedraEN{Department of Theoretical Computer Science and Mathematical Logic at Faculty of Mathematics and Physics}

% Jedná se o katedru (department) nebo o ústav (institute)?
\def\TypPracoviste{Katedra}
\def\TypPracovisteEN{Department}

% Vedoucí práce: Jméno a příjmení s~tituly
\def\Vedouci{Mgr. Martin Pilát, Ph.D.}

% Pracoviště vedoucího (opět dle Organizační struktury MFF)
\def\KatedraVedouciho{Katedra teoretické informatiky a matematické logiky}
\def\KatedraVedoucihoEN{Department of Theoretical Computer \\Science Mathematical Logic}

% Studijní program a obor
\def\StudijniProgram{Informatika}
\def\StudijniObor{Programování a Softwarové Systémy}

% Nepovinné poděkování (vedoucímu práce, konzultantovi, tomu, kdo
% zapůjčil software, literaturu apod.)
\def\Podekovani{%
Rád bych poděkoval vedoucímu mojí bakalářské práce panu Mgr. Martinovi Pilátovy, Ph.D. za podměty a odbornou pomoc při průběhu vypracování bakalářské práce. Dále bych rád pronesl poděkování své rodině za podporu při celém studiu.
}

% Abstrakt (doporučený rozsah cca 80-200 slov; nejedná se o zadání práce)
\def\Abstrakt{%
Robotická hejna se často díky svým dobrým vlastnostem používají při řešení rozličných úkolů. Řada prací zabývající se problematikou používá pro optimalizaci řízení robotických hejn evoluční algoritmy, ovšem z pravidla hejna tvoří pouze homogenní jedinci. Tato práce se zaměřuje na použití evolučních algoritmů v případě heterogenních robotických hejn. Pro nalezení optimalizující metody byla implementována simulace 2D prostředí, které umožňuje tvorbu vlastních scénářů pro robotická hejna a také umožňuje použití na řešení těchto scénářů evolučních algoritmů. Navržená metoda podúkolů používající diferenciální evoluci a evoluční strategie byla otestována ve třech rozličných scénářích. 
}
\def\AbstraktEN{%
Robotic swarms are often used for solving different tasks. Many articles are focused on generating robot controllers for swarm behaviour using evolutionary algorithms. Most of them are nevertheless considering only homogenous robots. The goal of this thesis is to use evolutionary algorithms for behaviours of heterogeneous robotic swarms. A 2D simulation was implemented to explore swarm controller optimization methods with the ability to create custom scenarios for robotic swarms. We tested differential evolution and evolution strategies on three different scenarios.}

% 3 až 5 klíčových slov (doporučeno), každé uzavřeno ve složených závorkách
\def\KlicovaSlova{%
hetorogenní robotická hejna, 
evoluční algoritmy, 
evoluční strategie, 
diferenciální evoluce
}
\def\KlicovaSlovaEN{%
heterogeneous robotic swarm,
evolutionary algorithms,
evolution strategies,
differential evolution 
}

%% Balíček hyperref, kterým jdou vyrábět klikací odkazy v PDF,
%% ale hlavně ho používáme k uložení metadat do PDF (včetně obsahu).
%% Většinu nastavítek přednastaví balíček pdfx.
\hypersetup{unicode}
\hypersetup{breaklinks=true}

%% Definice různých užitečných maker (viz popis uvnitř souboru)
\include{makra}

%% Titulní strana a různé povinné informační strany
\begin{document}
\include{titulka}

%%% Strana s automaticky generovaným obsahem bakalářské práce

\tableofcontents

%%% Jednotlivé kapitoly práce jsou pro přehlednost uloženy v samostatných souborech
\chapter*{Úvod}
\addcontentsline{toc}{chapter}{Úvod}

\title{Úvod}
Využití robotických hejn (robotic swarms) patří mezi rentabilní metody pro řešení složitějších úkolů. Existuje řada studií potvrzujících, že velký počet jednoduchých robotů dokáže plně nahradit komplexnější jedince. Dostatečná velikost hejna umožní řešení úloh, které by jednotlivec z hejna provést nesvedl. Navíc robotické hejno přináší několik výhod, díky kvantitě jsou odolnější proti poškození či zničení, neboli zbytek robotů pokračuje v plnění cílů. Dále výroba jednodušších robotů vychází levněji než komplexní jedinců, což přináší nezanedbatelnou výhodu pro práci v nebezpečném prostředí. Hejno také může pokrývat vícero různých úkolů než specializovaný robot, který bude při plnění úkolů, lišících se od typu úloh zamýšlených při konstrukci, mnohem více nemotorný a nejspíše pomalejší. Hejno pokryje větší plochu při plnění úkolů. 
\par
Existuje mnoho aplikací robotických hejn, vetšinou se používají v úlohách týkajících se průzkumu a mapování prostředí, hledání nejkratších cest, nasazení v nebezpečných místech \citep{swarmApp}. Jako příklad můžeme uvést asistenci záchranným složkám při požáru \citep{fireRobots}. Mnoho projektů zabývající se řízením robotického hejna se inspiruje přírodou, používá se analogie s chováním mravenců a jiného hmyzu \citep{PheroRobot}. Objevují se i hardwarové implementace chování hejn, zmiňme projekty Swarm-bots \citep{swarmBots}, Colias \citep{Colias}  
\par 
Elementárnost senzorů i efektorů jednotlivých robotů vybízí k použití evolučních algoritmů, jelikož prostor řešení je rozlehlý a plnění cílů lze vhodně ohodnotit. Vzniklo několik vědeckých prací popisující problematiku tohoto tématu \citep{ENovel} \citep{geneticSwarm}.
\section*{Cíl práce}
Všechny zmíněné práce používají pro tvorbu řídicích programů evoluční algoritmy (EA) a pracují pouze s homogenními hejny. Cílem této práce je vyzkoušet využití EA na generování chování hejna heterogenních robotů, tedy skupiny agentů, ve které se objevuje několik druhů jedinců a společně plní daný úkol. V rámci práce byl vytvořen program pro simulaci různých scénářů. Pro otestování jejich úspěšnosti v rámci EA, byly zvoleny 3 odlišné scénáře, ve kterých se objevují 2-3 druhy robotů.
\section*{Struktura práce}
Rozdělení práce je následující. První kapitola je věnována obecnému úvodu do tématiky evolučních algoritmů, kde se podrobněji věnuji evolučním strategiím a diferenciální evoluci, protože oba tyto postupy implementuji v programu pro řešení scénářů. Druhá kapitola se zabývá představením robotického hejna, základním principům a několika konkrétnějším aplikacím. Ve třetí kapitole je nastíněno fungování simulátory, přiložené dokumentace obsahuje podrobnější informace o implementaci simulátoru. Všechny provedené experimenty se všemi detaily obsahuje čtvrtá kapitola.
\chapter{Evoluční algoritmy}
\section{Historie}
Začněme pohledem do historie Evolučních algoritmů na základě knih \citep{MitchellBook} a \citep{eibenIntro}. Darwinova myšlenka evoluce lákala vědce už před průlomem počítačů, Turing vyslovil myšlenku \textit{genetického a evolučního vyhledávání} už v roce 1948. V 50. a 60. letech nezávisle na sobě vznikají 4 hlavní teorie nesoucí podobnou myšlenku. Společným základem všech teorií byla evoluce populace kandidátů na řešení daného problému a jejich následná úprava způsoby hromadně nazývány jako genetické operátory, například mutace genů, přirozená selekce úspěšnějších řešení. \par 
Rechenberg a Schwefell (1965, 1973) představuje \textit{Evoluční strategie}, metoda optimalizující parametry v reálných číslech, jejich použití pro letadlová křídla. Fogel, Owens, Walsh zveřejňují \textit{evolutionary programming}(evoluční programování), technika využívající k reprezentaci kandidátů konečný automat(s konečným počtem stavů), který je vyvíjen mutací přechodů mezi stavy a následnou selekcí. \textit{Genetické algoritmy} vynalezl Holand v 60. letech a následně se svými studenty a kolegy z Michiganské Univerzity implementoval, oproti ES a EP nebylo hlavním cílem formovat algoritmus pro řešení konkrétních problémů, ale přenos obecného mechanismu evoluce jako metody aplikovatelné v informatickém světě. Princip GA spočívá v transformaci populace chromozonů(př. vektor 1 a 0) v novou populaci pomocí genetických operátorů křížení, mutací a inverze. V 1975 v knize \textit{Adaptation in Natural and  Artificial Systems} \citep{HolandBook} definoval genetický algoritmus jako abstrakci biologické evoluce spolu s teoretickým základem jejich používání. Ovšem někteří vědci používají pojem GA i ve významech hodně vzdálených původní Holandově definici. K sjednocení jednotlivých přístupů přispěl v 90. letehc Koza, dále jsou všechny zahrnuty jako oblasti \textit{Evolučních algoritmů}. Dnes existuje řada konferencí a odborných časopisů sdružující pracovníky zabývající se touto oblastí. Zmiňme ty větší z nich, co se týče konferencí: 
\href{http://gecco-2017.sigevo.org/index.html/HomePage}{GECCO}, \href{http://www.ppsn2016.org/conference}{PPSN}, 
\href{http://www.cec2017.org/}{CEC}, 
\href{http://www.evostar.org/2018/}{EVOSTAR}, 
časopisy: 
\href{http://www.mitpressjournals.org/loi/evco}{Evolutionary Computation}, 
\href{http://ieeexplore.ieee.org/xpl/RecentIssue.jsp?reload=true&punumber=4235}{IEEE Transactions on Evolutionary Computation}, 
\href{http://www.springer.com/computer/ai/journal/10710}{Genetic Programming and Evolvable Machines},
\href{https://www.journals.elsevier.com/swarm-and-evolutionary-computation/}{Swarm and Evolutionary Computation}
\section{Obecné principy}.
%%% Druhá kapitola

\chapter{Robotický swarm}
Myšlenka robotického swarmu pochází podobně jako u genetických algoritmů z inspirace matkou Přírodou. Podle souhrnu \citep{swarmRobotic} popíši základní myšlenku RS. Pro robotický swarm se také používá výraz rojová robotika či robotický roj, v angličtině je známý pod pojmem swarm robotics. 
\section*{Základní vlastnosti}
Motivací pro použití RS může být chování živočichů na Zemi. Zaměříme se na skupiny živočichů jako jsou mravenci, včely, ryby dokonce i některé savce. Pokud bychom vložili do prostředí jednotlivce z některé ze zmíněných skupin, nebude schopen konkurovat nepřátelskému prostředí a nejspíše příliš dlouho nepřežije. Na druhou stranu, když budeme uvažovat celé společenství, tak se nám ze slabého jedince stane velmi adaptivní, odolný a rychle se vyvíjející roj. Podobnému účinku bychom se chtěli přiblížit v RS. Pro relativně jednoduchého robota, který není schopen plnit obtížný úkol, se pokusíme použít vícero robotů stejného typu, kteří společně zadaný úkol vyřeší. Navíc chceme těžit ze všech výhod hejna. \par. 
\par
Jako nejčastější výhody RS oproti jednomu robotovi se nejčastěji uvádějí:
    \begin{enumerate}
        \item Paralelita - Díky malé ceně jedince, si můžeme dovolit velkou populaci jedinců. Malou cenou jedince v ES myslíme, že se jedná o jednoduchého robota s nízkou pořizovací cenou. V kontextu živočichů můžeme uvažovat množství energie, jídla pro tvorbu takového jedince. Velká populace nám umožňuje řešit vícero úkolů naráz, také na velké ploše. Zvláště pro vyhledávací úkoly ušetříme nemalé množství času. 
        \item Škálovnatelnost - Změna velikosti populace hejna neovlivní chování ostatních jedinců. Samozřejmě plnění úkolu bude rychlejší či pomalejší, ale původní hejno bude stále plnit původní úkol. Tím pádem můžeme celkem snadno upravovat velikost populace bez větší obtíží. V přírodě můžeme pozorovat, že smrt  jednotlivých mravenců-dělníků znatelně neovlivní práci celého mraveniště. Nově narození mravenci se mohou vydat do práce, zatímco zbytek mraveniště nemění činnost. 
        \item Houževnatost - Související se škálovatelností, jen v tomto případě máme na mysli necílenou změnu populace. Jako v předchozím příkladu, u smrti mravenců, část robotů ES může selhat z rozličných důvodů, zbytek hejna však bude pokračovat k cíli, i když ve výsledku jim bude jeho dosáhnutí trvat o něco déle. To se nám může vyplatit v nebezpečných prostředích. 
        \item Ekonomické výhody - Cena návrhu a konstrukce jednoduchých hejn robotů vyjde většinou levněji než jeden specializovaný robot schopný uspokojit stejné požadavky. V dnešním světě vychází výroba ve velkém množství mnohem levněji než tvorba jednoho drahého konkrétního robota.
        \item Úspora energie - Díky menší velikosti a složitosti jednotlivých robotů vyžadují mnohem menší množství energie. To má za důsledek, že si u nich můžeme dovolit energetickou rezervu na delší čas. Navíc když je pořizovací cena jednoho robota menší než náklady na dobití, můžeme díky škálovatelnosti pouze připojit nové roboty, což u drahého robota jde málokdy. 
        \item Autonomie a decentralizace - V kontextu RS musí každý jedinec hejna jednat autonomně, jedinci nejsou řízeni žádnou autoritou. Takže umí pracovat i při ztrátě komunikace. Opět se vychází z chování živých organismů. Pokud se chovají jedinci hejna dostatečně kooperativně, mohou pracovat bez centrálního řízení, důsledkem toho se stává celé hejno ještě flexibilnější a odolnější, hlavně v prostředích s omezenou komunikací. Navíc hejno mnohem rychleji reaguje na změny. 
    \end{enumerate}
\par 
Mimo RS existuje i řada jiných přístupů, které se inspirovaly životem hejn v přírodě. Občas jsou zaměňovány za RS, nejčastěji se jedná o multi-agentní systémy a senzorové sítě (sensor networks). V následující tabulce jsou popsány jejich nejklíčovější vlastnosti. \par
\begin{center}
    \begin{table}[h] \resizebox{\textwidth}{!}{%
    \begin{tabular}{l l l l l @{\hspace{1.5cm}}D{.}{,}{3.2}D{.}{,}{1.2}D{.}{,}{2.3}}
            \toprule
             & \textbf{Robotická hejna} & \textbf{Multi-robotické systémy} & \textbf{Senzorové sítě} & \textbf{Multi-agentní systémy} \\
            \textbf{Velikost populace }& Variabilní ve velkém rozsahu & Malá & Fixní & V malém rozsahu \\
            \textbf{Řízení} & decentralizované a autonomní & centralizované & centralizované & centralizované \\
            \textbf{Odlišnosti} & většinou homogenní & většinou heterogenní & homogenní & homogenní nebo heterogenní \\
            \textbf{Flexibilita} & vysoká & nízká & nízká & střední \\
            \textbf{Škálovnatelnost} & vysoká & nízká & střední & střední \\
            \textbf{Prostředí} & neznámé & známé nebo neznámé & známé & známé\\
            \textbf{Pohyblivost} & ano & ano & ne & vyjímečně\\
            \hline 
    \end{tabular}}
	\caption{Porovnání systémů s více agenty}
    \end{table}
    \end{center}
    Jednotlivé systémy s více agenty se také liší svojí aplikací. Robotická hejna se nejčastěji používají ve vojenských, nebezpečných úkolech a také pro řešení ekologických katastrof. Multi-robotické systémy potkáváme v transportních, skenovacích úkolech, dále pro řízení robotických fotbalových hráčů. Oproti tomu nejčetnější využití senzorových sítí zasahuje do medicínské oblasti, ochrany životního prostředí. Multi-agentní systémy zase nejvíce zasahují do řízení síťových zdrojů a distribuovaného řízení. 
\section{Použití}
Existuje několik vědeckých prací, které studují a navrhují použití RS v reálném nasazení. 
\par 
Některé jsem zmínil už v úvodu této práce, jako například hasičům asistující roboty \citep{fireRobots}.  Zde si robotické hejno klade za cíl usnadnit a podpořit navigaci lidem v nebezpečném prostředí. Jejich využití je ilustrováno záchranou misí ve velkém skladišti. Hasiči mají díky kouři velmi omezenou viditelnost. Tím pádem se lokalizace přeživších, epicenter požárů a další důležitých bodů stává obtížným a zdraví ohrožujícím úkolem. Nezřídka se stává, že zasahující hasič zahyne, protože se v hustém dýmu v objektu ztratil. Robotické hejno tedy může prozkoumat celý prostor před vlastním zásahem. Při zásahu ještě asistovat hasičům při orientaci v prostoru. 
\par
Robotická hejna se také ukázala jako užitečná u ekologický pohrom. Španělští vědci testovali jejich použití při úniku ropy \citep{oilSwarm}, či při hledání centra radiace \citep{radiationSwarm}. 
\par 
V prvně zmíněném příkladu autoři mapovali znečištění mořské vody. Dokonce při plavbách bez defektů se do moře uvolňuje palivo, ropa a další nebezpečné látky. Očekává se, že s rostoucí námořní dopravou se rozrostou tyto lokální znečištění ve vážnou hrozbu. Aktuální systémy monitorující znečištění při katastrofách tankerů jsou pro toto využití příliš drahé. Autoři proto navrhují použití hejna dronů, které bude dokumentovat velké vodní plochy a bude schopno odhalovat případné nebezpečí a větší koncentrace cizích látek. Dokonce budou moci na základě získaných informací sledovat znečisťovatele. 
\par
Po jaderné katastrofě se stává explorace zamořené oblasti  v podstatě nemožným úkolem pro lidské průzkumníky. Právě monitorování radiací postiženým územím se stala motivací pro práci \citep{radiationSwarm}. Ve zmiňované práci se autoři soustředí na porovnávání autonomních robotických hejn a RS komunikujících s člověkem. V rámci výsledků ukazují, výhody použití robotického hejna pro hledání centra radiace a také že RS interagující s lidmi dosahují lepších výsledků.
\par
Několik prací nezůstalo pouze u simulací a také využívali RS u fyzických robotů. Hlavní motivací pro práci \citep{aquaticRobots} byl fakt, že většinou se řízení RS vytváří  a hlavně testuje pouze v uzavřeném a simulovaném prostředí. Tvůrci se rozhodli pro reálné použití na moři, kde nemohou prostředí jakkoli ovládat či kontrolovat. Snaží se tím ukázat, že i přes šumy a neočekávané situace, RS je stabilní a použitelné pro aplikaci ve skutečném světě. Celé hejno se skládalo z deseti robotů. Jednalo se o malé lodičky s délkou přibližně 60 cm a nízkou pořizovací cenou okolo 300 euro. Každý robot byl  vybaven GPS, WIFI, kompasem. Chování bylo připraveno pomocí evolučních algoritmů v simulaci, konkrétně autoři používají neuroevoluci NEAT, jejich simulace obsahovala 4 podúkoly: navádění, shlukování, rozptylování a monitorování prostředí. Poté bylo stvořené řízení ohodnoceno na vodní ploše. Prezentované výsledky vypadají velmi slibně, chování a úspěšnost řízení se velmi blíží mezi simulací a reálným nasazením. Také potvrzují přítomnost výhodných vlastností ze simulaci v reálném světě, jedná se o robustnost, flexibilitu, škálovatelnost. V neposlední řadě také úspěšnost skládání jednoduchých podúkolů do komplexního chování v rámci hejna, které řeší složitý hlavní cíl.  
\section{Řízení robotických swarmů}
Chování swarmů se řadí mezi velmi obtížné úkoly pro svět informatiky. Pro reprezentaci chování se využívá \textit{neuronových sítí}, které se optimalizují pomocí nastavování vah jednotlivých perceptronů, neboť se jedná o velký prostor vstupních informací ze senzorů a prostor pro interakci s prostředím je taktéž velmi rozsáhlý. Přímé prohledávání takto obřího prostoru nepřichází v úvahu, proto v poslední době získávají na oblibě evoluční algoritmy. Mezi nejvíce používané patří evoluční strategie, či genetické programování. \par 

\subsection*{Genetické programování a stromy chování}
V práci \uv{Evolving behaviour trees for Swarm robotics} \citep{Jones2018} se autoři zaměřují na využití genetického programování pro vytvoření chování robotického hejna. Pro řízení hejna navrhují vcelku zajímavé využití stromů chování (SC)(behaviour tree), které mají uplatnění především v herním průmyslu pro akce charakterů, které nejsou ovládány hráčem. Jako optimalizační algoritmus zvolili genetické programování. 
\par
Strom chování je strom, jehož listy interagují s prostředím, vnitřní vrcholy spojují tyto akce dohromady a tvoří rozhodovací a závislostní pravidla. Celý strom je vyhodnocován v pravidelných intervalech, v práci se značí jako \textit{tick}. Opírají se o článek \citep{shoulson2011parameterizing}, kde bylo ukázáno, že SC může plnohodnotně reprezentovat konečné automaty, dokonce i když budeme používat pravděpodobností konečné automaty. Jako jedince z hejna zvolili Kilobot, které byl představen Rubensteinen v \citep{Kilobots}. Kiloboti se pohybuji pomocí dvou vibračních motorů, komunikují přes infračervený kanál, v prostředí se orientují pomocí foto detektoru a signalizují pomocí LED diod s barevným spektrem. Tabulka \ref{tab02:behatree} ukazuje, jak byla zjednodušena komunikace s efektory robota a nad nimi optimalizováno SC. 
\par
\begin{table}[h]\resizebox{\textwidth}{!}{%
		\begin{tabular}{l l l l l @{\hspace{1.5cm}}D{.}{,}{3.2}D{.}{,}{1.2}D{.}{,}{2.3}}
        \toprule
        Efektor/Senzor & Read or Write & Popis \\
        \midrule
        motor & W & vypnut, vřed, vlevo, vpravo \\
        přídavná paměť & R\&W & libovolná hodnota \\
        vysílač signálu & R\&W & vysílá při větší hodnotě než 0.5\\
        přijímač signálu & R & 1 pokud přijímá signál \\
        detektor potravy & R  & 1 pokud světelný snímač vidí potravu\\
        nosič jídla & R & 1 pokud nese jídlo \\
        hustota robotů & R & hustota Kilobotů v blízkosti \\
        $\delta hustota$ & R & změna v hustotě \\
        $\delta vzdálenost_{potrava}$ & R & změna ve vzdálenosti k potravě \\
        $\delta vzdálenost_{hnízdo}$ & R & změna ve vzdálenosti k hnízdu \\ 
        \bottomrule
    \end{tabular}}

	\caption{Parameterizing behavior trees, Motion in Games - podoba stromů}
		\label{tab02:behatree}
\end{table}
\par
Akce z tabulky \ref{tab02:behatree} pak odpovídají listům v SC, vnitřní vrcholy mohou být kompoziční: \textit{seqm}, \textit{selm}, \textit{probm} a tyto mohou mít 2 až 4 syny. Informace procházející mezi vrcholy mohou být následujícího druhu \textit{success}, \textit{failure}, \textit{running}. Zpracovávají informace následujícím způsobem posílají tik do každého syna dokud od nějakého nepřijde hodnota \textit{failure} nebo tik proběhne na všech synech. Pokud proběhne úspěšně tik u všech synů vrací \textit{success}, \textit{failure} jinak. Oproti tomu \textit{selm} vysílá tik, dokud mu nějaký syn nevrátí hodnotu \textit{success} nebo všichni synové provedli tik, pokud se nevrátí jediná hodnota \textit{success}, tak vrací \textit{failure}, v opačném případě \textit{success}. Od obou se liší \textit{probm}, ten s danou pravděpodobností vybere jednoho syna a vrátí jeho odpověď. Vrchol, který má alespoň jednoho syna se statusem \textit{running} vrací stejnou hodnotu. \par
Vrcholy jen s jedním synem patří do jedné z následujících kategorií: \textit{repeat}, \textit{succeeded}, \textit{failed}. Vrchol \textit{repeat} vrací tik svým synům, s daným počtem pokusů, dokud nedostane hodnotu \textit{success}. Následující dva vrcholy vrací konstantní odpověď na tik dle svého jména, i přesto pošlou tik svému následníkovi. \par
Poslední skupinu vrcholu tvoří tzv. akční vrcholy \textit{ml}, \textit{mr}, \textit{mf}, což ve stejném pořadí jsou: zatoč vlevo, zatoč vpravo, jeď kupředu. Vrcholy pohybu při prvním tiku vrací \textit{running} při druhém \textit{success} K akčním vrcholům také patří \textit{if}, který slouží k porovnávání jeho dvou synů, pokud porovnání platí vrací \textit{success}. Poslední z akčních vrcholů je \textit{set}, který nastavuje danou hodnotu synům. 
\par
V prostředí jsou s konstantní frekvencí prováděny tzv. \uv{update} cykly. Každý cyklus se skládá ze třech částí po sobě jdoucích částí.  
\par
\begin{enumerate}
    \item Spočítají se hodnoty v synech z vysílaných signálů a prostředí. 
    \item Na SC je proveden tik, což čte a zapisuje hodnoty do synů. 
    \item Pohybové motory jsou aktivovány a vysílání je upraveno, oboje dle zapsaných hodnot do synů.
\end{enumerate}
\par 
Jako zátěžový test používají obvyklý scénář, který spočívá v hledání potravy a jejím odvážení zpět do hnízda. Fitness se hodnotí podle vzdálenosti doneseného jídla, čím blíže k hnízdu tím lépe. Jako optimalizační metodu autoři vybrali genetické programování a používají DEAP knihovnu \citep{deap}. \
Celá populace velikosti $n_{pop}$ je ohodnocena fitness, každý jedinec se hodnotí podle 10 simulací, každá simulace má jinou startovní konfiguraci. Roboti startují vždy na poli o velikosti 5x5, jejich orientace je vybírána náhodně. Simulaci běží 300 sekund, update cyklus pro vnímání prostředí má frekvenci 8 Hz a u ovladačů s 2Hz. Používané genetické programování implementuje elitimus přenáší $n_{elite}$ nejlepších jedinců do další generace, zatímco zbylá část je zvolena  turnajovou selekcí s velikostí $t_{size}$. Křížící (rekombinační) operátor, který kříží části stromů, je aplikován s pravděpodobností $p_{xover}$ na všechny páry vybrané turnajovou selekcí. Na vzniklé páry se aplikují 3 mutační operátory. \par
\begin{enumerate}
    \item S pravděpodobností $p_{mutu}$ je náhodný vrchol stromu vyměněn za nový náhodně vytvořený 
    \item S pravděpodobností $p_{muts}$ je náhodná větev stromu a je  vyměněna za náhodně zvolený terminál (vyskytující se na této větvi)
    \item S pravděpodobností $p_{mutn}$ je náhodný vrchol vyměněn za nový, ale se stejným počtem argumentů
    \item S pravděpodobností $p_{mute}$ je náhodná konstanta vyměněna za jinou náhodnou hodnotu
\end{enumerate}
Krom simulace 25 nezávislých běhů evoluce, také byly otestovány algoritmy na reálných strojích. Vytrénované chování bylo otestováno 20 běhy s rozdílnou startovací pozicí a ohodnoceni stejnou fitness. 
\par
Výsledky simulace byly více než uspokojivé v simulační části si hejno vedlo o trochu lépe 0.075 z maximální hodnoty 0.12 (minimum 0) a co se týče reálného nasazení vygenerovaného chování 0.058. Což opravdu není velký rozdíl, když přihlédneme k tomu, že evoluce probíhala čistě na simulační rovině.

\subsection*{Genetické algoritmy a neuronová síť}
Cagri a Yalcin používají ve své práci \citep{yalcin2008evolving} práci neuronových sítí místo SC a pro nastavení vah genetického algoritmu. Shodují se s Winfieldem a jeho kolegy, že evoluce mnoho chování robotického hejna přináší zajímavých strategií, která mohou být mnohem komplexnější než explicitně vytvořené chování. Popisují však také obtížnosti použití evoluce, zvláště volbu evolučního algoritmu a efektivnost celého výpočtu. 
\par
Využívají již existujícího simulátoru Cobot2D, pro všechny experimenty bylo použita mapa bez překážek velikosti 400x400. Roboti se pohybují pomocí dvou-kolečkových motorů, orientují se 4 infračervenými senzory a 4 zvukovo-směrovými senzory, v jejich středu je umístěn všesměrový zvukový vysílač. Vysílače mají pevný rozsah kruhu vysílání a dynamickou sílu. Zvukové senzory se rozhodují pouze podle signálů, jejichž vysílače spadají do $90^\circ$ výseče od senzoru a také jejich vzdálenost musí být menší než daná konstanta. Síla signálu se zmenšuje směrem od vysílače a senzory vrací součet sil signálů. Infračervené senzory skenují úsečku dané velikostí a vrací vzdálenost k nejbližšímu průsečíku. V rámci simulace jsou generovány náhodné šumy pro každou interakci s prostředím, aby se simulace přiblížila co nejvíce reálnému nasazení. Pro ovládání robota byla navržena neuronová síť, která má 8 vstupů (4 pro infra-senzory a 4 pro zvukové) a 3 výstupy a není zde žádná skrytá vrstva. Nákres robota a jeho ovládací neuronové sítě můžete vidět na obrázku. \par
\begin{figure}[h]\centering
\includegraphics[scale=0.5]{../img/Cobot.png}
\caption{Cobot2D - nákres robota, zdroj : \citep{yalcin2008evolving} }
\end{figure}
\textbf{Fitness funkce}: První použitá $fitness_1$ z tohoto experimentu je obrácená hodnota průměrné vzdálenosti do středu robotické skupiny. 
\par
\begin{center}
\textbf{$fitness_1 = 1/(1/n\sum\limits_{r=1}^{n} d_{rc}) $},
\end{center}
\par 
Kde $n$ je počet robotů v robotické skupině, $r$ je robotův index, $d_{rc}$ je euklidovská vzdálenost mezi $r$ a středem robotické skupiny $c$. 
\par
Druhá $fitness_2$ používá metodu \textit{inverse of hierarchical social entropy} \citep{balch2000hierarchic}. Tato metoda počítá kompaktnost skupiny, tím že hledá každou možnou skupinku (cluster) pomocí změn maximální vzdálenosti $h$ mezi jedinci ze stejného clusteru. Přidávají ještě rozšíření od \textit{Shannon's information entropy}, jenž používá pevné $h$. Toto rozšíření je definováno: 
\par 
\begin{center}
\textbf{$H(h)=-\sum\limits_{k=1}^{M} p_k log_2(p_k)$},
\end{center}
\par 
kde H se nazývá entropie, $p_k$ je proporce jedince z clusteru $k$, $M$ je počet clusterů pro dané $h$. Konečně celý předpis daný Balchem vypadá následovně: 
\par
\begin{center}
\textbf{$fitness_2 = \int_{0}^{\infty} \frac{1}{H(h)dh}$}.
\end{center}
\par 
Použití neuronových sítí a genetického algoritmu se ve výsledku ukázalo jako vhodný prostředek pro učení robotického hejna, neboť se vygenerované chování obstojně shlukuje do úzkých skupin. Autoři dále definují další tzv. cost funkci pro měření úspěchu nalezených chování, aby mohli porovnávat funkce fitness. A $fitness_2$ se ukazuje jako účinější. 
\subsection*{Evoluční strategie a neuronová síť}
V článku \textit{Self-organised path formation in a swarm of robots} \citep{sperati2011self} aplikují pro řízení robotických hejn evoluční strategie. Jako cíl si článek klade problém průzkumu a navigace v neznámém prostředí v kontextu robotických hejn. Experiment, který měl otestovat uvedené vlastnosti robotického hejna, spočíval v co nejrychlejším přesunu celého hejna mezi dvěma prostory v neznámém prostředí. 
\par 
Pro simulaci bylo využito upravené verze OS Evorobota a jako model jedince z hejna e-puck robot \citep{mondada2009puck}. Tento robot se pohybuje pomocí dvou-koleček, má 8 infračervených senzorů, navíc jeden infračervený senzor na povrch a jeden rozpoznávající barvy vpředu (v tomto případě černobílé prostředí). Navíc mu byla přidělána LED vpředu s modrou barvou a červenou vzadu, která může zapínat a vypínat dle potřeby, a také má snímač barev na vrchu. 
\par
Pro ovládání robota zvolili autoři neuronovou síť se 13 vstupy (8 pro infračervené senzory, 1 pro binární podlahový senzor (bílá vs. černá), 4 pro binární vizuální snímače), dále 3 pro skryté neurony a 4 výstupní neurony (2 ovládající kolečka, 2 aktivující přední a zadní led). Formou jsou podobné předchozím modelům robotů. 
\par
Fitness funkce je vyhodnocena po nasazení do robotů a provedení simulace, vlastní fitness je pak průměr z 15 běhů. Ve vyznačených místech se roboti nabíjí, což trvá daný čas a roboti s lepší efektivitou přesunů z jednoho místa do druhého stihnout cestu tam a zpět mnohem rychleji.
\par 
Výsledky prokazatelně ukazují úspěšné použití evolučních strategií na optimalizaci chování robotického hejna. Pro většinu prostředí dokázali najít efektivní řešení.



%%% Fiktivní kapitola s ukázkami tabulek, obrázků a kódu

\chapter{Simulátor}
Součástí bakalářské práce bylo také naprogramování simulátoru mapy, její vizualizace, implementace všech EA, vše v jazyce C\#. Veškeré informace o simulátoru a jeho součástech můžete najít v dokumentaci na přiloženém CD. V této kapitole popíši pouze spuštění, základní objekty simulátoru a aplikované optimalizace. \par 
\paragraph{Části simulátoru}
\begin{itemize}
	\item \textit{SwarmSimFramework.exe} - konzolová aplikace, která obsahuje kód pro EA optimalizaci a simulaci mapy
	\item \textit{SwarmSimVisu.exe} - program pro vizualizaci, implementovaný ve WPF a C\# 
\end{itemize}

\section*{Spuštění}
Pro spuštění optimalizace SwarmSimFramework.exe je potřeba soubor s konfigurací experimentu, soubory s konfigurací pro ES mají koncovku \textit{.es} a v případě DE \textit{.expe}. Jednotlivým EA také odpovídají očekávané argumenty. 
\begin{itemize}
	\item ES - \textit{SwarmSimFramework.exe -es soubor\_konfigurací.es}  
	\item DE - \textit{SwarmSimFramework.exe -de soubor\_konfigurací.de}. 
\end{itemize}
\par 
V rámci konfiguračních souborů můžeme nastavit charakteristiky mapy, ohodnocení jednotlivých složek fitness, specifikovat druh či počet robotů, název experimentu a název složky s výstupem. Výstupní složka obsahuje serializované nejlepší jedince vždy po 10 generacích, metadata pro graf (číslo generace, hodnoty fitness), serializované veškeré jedince po 100 generacích. V případě ES mají jedinci svou vlastní podsložku z implementačních důvodů (paralelismus). Populace z poslední generace je ještě serializována do složky spuštění. Do konzole, pak program vypisuje základní informace o aktuální generaci.
\par 
Pokud si chceme prohlédnout některé z chování vizuálně, spustíme program SwarmSimVisu.exe. Kde je připraveno grafické rozhraní, kde pomocí tlačítek navolíme vybraný scénář, nastavení mapy a soubory se serializovaným řízením robota. Poté je možné danou simulaci spouštět a zastavovat, při zastavené simulaci pravým klikem na robota lze zobrazit jeho podrobné informace.

\section*{Optimalizace}
Simulace mapy patří mezi časově nejvíce náročnou část. Pro její zrychlení jsem použil paralelního programování, jeho přístup se liší podle EA. Za použití profileru jsem nalezl, že nejvíce času simulace zabralo počítání jednotlivých průsečíků pro senzory, efektory a entity samotné. Pro optimalizaci počtu průsečíků jsem implementoval vlastní datovou strukturu \textit{SpatialHash}. 
\par
Rozdělení ES na části, které jsou paralelně zpracované, je vcelku přirozené, protože každý jedinec používá pro tvoru potomka mutace odvozeného z sebe samotné. Tím pádem můžeme od sebe oddělit jednotlivé jedince a nemusí sdílet žádnou paměť. U DE bylo rozdělení poněkud složitější, protože v rámci běhu vybírá náhodně z celé populace. Rozhodl jsem se, že oddělím simulace mapy, která probíhá pro nově vzniklého potomka a na její základě se hodnotí jeho fitness. Takže pro každého člena původní populace vznikne nezávislý proces, který vybere z aktuální populace 3 jedince ( pouze čtení) a pak nově vzniklého jedince vloží do thread safe datové struktury (list z knihovny System.Collections.Concurrent). Po dokončení všech procesů se přejde k další generaci. \unsure{Nějaké větší detaily, CPU, stejný seed pro všechno, ale nemůžu ovlivnit pořádí volání random number} Použití paralelního zpracovaní znemožňuje přesné zopakování experimentů, zkoušel jsem experimenty pouštět na rozličných strojích a výsledky řádově odpovídají.  
\par 
SpatialHash rozděluje mapu na sít malých čtverečků. Každá entita (mimo roboty) je uvedena ve všech čtverečcích do kterých zasahuje. Pro počítání kolizí dostane potenciálně kolidující těleso od SpatialHash množinu obsahující všechny entity zasahujících do stejných čtverečků jako potenciálně kolidující těleso. Jedinou vyjímku tvoří roboti, neboť se často přesouvají po mapě a ve SpatialHash by neustále měnili pozici. Operace spojené s častým pohybem byly více náročné než počítat průsečíky se všemi roboty, takže roboti jsou uloženi v klasickém listu.
\section*{Použití}
Pro jednotlivé objekty na mapě jsem vytvořil obecný objektový návrh, který umožňuje jednoduché přidání vlastních objektů. Případně používat pouze simulaci na mapě pro jiné účely než optimalizaci chování robotů. Návrh nových evolučních algoritmů. \unsure{Jak moc má todle má být podrobný?} 

\redo{Ještě něco přidat?} 
\redo{Nějaký obrázky? }
%%% Kapitoly
\chapter{Experimenty}
\label{chap:experimenty}
Všechny práce zmíněné v úvodní kapitole používají evoluční algoritmy k vytvoření řízení chování homogenního robotického hejna, tzn. s jedním druhem robotů. V následující kapitole podrobně popíši postup hledání optimální chování pro heterogenního hejna. Optimalizaci jsem navrhl a otestoval na třech rozličných scénářích. 
\par
\subsubsection{Pracovní názvy scénářů:}
\begin{enumerate}
	\item Wood Scene - zpracování dřeva
	\item Mineral Scene - přetvoření minerálů na palivo 
	\item Competitive Scene - soubojový scénář
\end{enumerate}

Hlavní motivací při tvorbě scénářů bylo vytvořit obtížnější úkoly než se obvykle používají jako například: shlukování, vyhýbání překážkám, atp. Navrhnout je natolik komplexně, aby nebylo možné, že část hejna se nebude podílet na jeho plnění. Také jsem volil scénáře, aby se přiblížily situacím z reálného světa. Každému z nich jsem věnoval samostatnou kapitolu, která zahrnuje popis hlavního úkolu scénáře, seznam robotů i s jejich senzory a efektory, způsob hodnocení fitness,rozdělení do podúkolů s průběhem fitness u ES a DE, vizualizaci a rozbor chování nejlepšího jedince.  
\par
Pro řešení problému jsem navrhl řadu postupů, proto v tomto odstavci zmíním ty nejvíce přímočaré a slibné, které se ovšem ukázaly  jako neúspěšné. V kapitolách zabývajícími se konkrétními scénáři už budu pouze popisovat jen konečné, úspěšné postupy.
\par
Nedostatečný se ukázal pokus provádět evoluci pro fitness hlavního úkolu scénáře. Konkrétně pro Wood Scene počet natěženého a uskladněného dřeva, pro Mineral Scene objem vytvořeného paliva, pro kompetitivní scénář zbylé body zdraví a udělené poškozený. Většina hodnocení náhodných chování byla rovna nule, proto EA nedostaly dostatek informací k vhodné exploraci a díky malé pravděpodobnosti vygenerování chování alespoň částečně řešící hlavní úkol nedocházelo ani k exploataci. Což mělo za důsledek neefektivní  DE a ES, takže ani jeden z EA nedošel k úspěšnému řešení. 
\par
Posun zaznamenala více obecná fitness i když sama o sobě také nedosáhla do kategorie úspěšných postupů. Do fitness jsem zahrnul i menší pozitivní znaky, které byly součástí hlavnímu úkolu. Například jsem záporně ohodnotil pokusy o pohyb končící kolizí, kladně počet nalezených entit či vhodných objektů v kontejnerech, aktuální stav paliva. Optimalizovaná chování opravdu zaznamenala posun. Ovšem oba EA obtížně hledaly cestu z lokálního optima a ve většině případů optimalizovali pouze jednoduché části úkolu. I přes přidávání složitějších matematických funkcí do fitness nebyly schopny dosáhnout uspokojivého řešení hlavního úkolu scénáře.  
\subsubsection*{Nějak pojmenovat}
Pro finální řešení jsem zvolil metodu, kterou nazývám metodou podúkolů. U každého scénáře podrobně popíši její průběh a nastavení, zde pouze nastíním základní  myšlenku. Rozdělil jsem hlavní cíl na několik menších podúkolů (metaúkolů). Každému z nich vytvoříme fitness funkci odpovídající nutné části hlavního cíle. Fitness metaúkolu jsem navrhoval, tak aby necílila na již optimalizované úkony a v každém podúkolu jsem se vždy soustředil pouze na jeden jednoduchý úkon. Díky tomuto principu jsem dosáhl mnohem vyšší odolnosti proti uvíznutí v lokálním minimu. Explorace se tímto procesem také zlepšila, protože hlavní cíl závisí na podúkolech a pokud bylo chování rozmanité a úspěšné, přenesly se tyto vlastnosti i dále. Poté jsem generaci úspěšnou v průzkumu optimalizoval na sbírání materiálů pouze požadované barvy a takto jsem rozděloval až k finálnímu úkolu scénáře. 
\par 
Každý robot má připojen paměťový slot, neboť roboti s nimi dosahovali ve všech úkolech znatelně lepších výsledků a pomáhaly vytvářet složitější chování. 
\section{Použité technologie}
Tuto kapitolu věnuji klíčovým technologiím, jenž jsem použil pro modelování řešeného problému a optimalizaci náhodných chování. Pro ovládání robotů jsem zvolil v poslední době velmi oblíbené \textit{neuronové sítě}, které se často používají v kombinaci s EA. Jednomu jedinci odpovídá jedna neuronová síť ovládající všechny části robota. Tyto neuronové sítě si lze představit jako vektor reálných čísel, což je vhodná reprezentace genotypu pro EA. 
\subsection*{Reprezentace Chování - neuronové sítě}
Pro reprezentaci jedinců v oblasti robotiky, rozpoznávání obrazů a dalších oblastí umělé inteligence se v poslední době používají nejčastěji neuronové sítě. Neuronová síť se strukturou podobá neuronovým sítím v mozku. Základní sítě se skládají z jednotlivých neuronů, které se v kontextu informatického světa nazývají \textit{perceptrony}. Samostatný perceptron je sám o sobě také neuronovou sítí, ale většinou se propojují do složitějších struktur. Perceptron lze definovat podle \citep{neuron} následovně.
\begin{definice}[Perceptron] Perceptron je funkce z $\mathbb{R}^n \rightarrow \mathbb{R}$, která je dáná následujícím předpisem: 	$Y = S(\Theta + \sum_{i=0}^{n} w_i x_i)$, kde pro $i \leq n$ $x_{i}$ je $ítý$ prvek vstupního vektoru, $w_{i}$ se označuje jako váha a většinou $w_{i} $ se bere z $\mathbb{R}$.  $\Theta$ se nazývá práh (bias) a slouží jako váha s konstantním vstupem 1.  $S(X)$ je aktivační funkce: $\mathbb{R} \rightarrow \mathbb{R}$ a $Y$ se obvykle označuje jako výstup perceptronu.
\end{definice}
\textit{Jednovrstvou neuronovou sítí} pak myslíme n perceptronů, tedy funkci $\mathbb{R}^{n} \rightarrow \mathbb{R}^{n}$, kde $ítou$ složku výstupního vektoru dostaneme aplikací funkce odpovídající $ítému$ perceptronu na vstupní vektor.
\improvement{Přidat část o aktivačních funkcích}
\par
Pokud zapojíme z výstupu jednoho perceptronu na vstup jiného, vznikne \textit{vícevrstvá neuronová sít}. Což znamená, že podmnožiny výstupů z první vrstvy neuronů neurčují přímo výstup, ale jsou opět zvoleny jako vstupní vektory pro další jednovrstvou neuronovou síť. Tímto postupem můžeme vytvářet velmi komplexní struktury.
\par
Skrytá vrstva (hidden layer) je taková jednovrstvá neuronová síť, jejíž výstup(resp. vstup) je pouze vstupem(resp. výstup) jiných perceptronů.
\par
\improvement{přeformulovat??}
Pro mé účely se jsem testoval řadu různých variant neuronových sítí, ale nejvíce se mi osvědčilo následující nastavení, které poskytovalo uspokojivé výsledky a rozumné časové nároky. 
\par 
Jako aktivační funkce jednotlivých perceptronů se mi nejvíce osvědčila často používaná funkce hyperbolického tangentu se změněným oborem hodnot pro konkrétní výstup.
\par
V rámci testovaní jsem zvolil jednoduchou architekturu jednovrstvé neuronové sítě, což se ukázalo jako dostatečné pro uspokojivé řešení jednotlivých scénářů. Jejich architektura je následující. Pro každé reálné číslo, které očekává robot jako vstup pro efektor, byl připojen perceptron do kterého vstupuje vektor reálných čísel odpovídající vektoru všech hodnot přečtených ze senzorů.  Pro dosažení lepších řešení by zde bylo možné nasadit NEAT algoritmus či hledat více specifičtější architektury.Případně vyzkoušet vliv vícero vrstev. \improvement{Přidat do diskuze}
\subsection*{Evoluční algoritmy}
Neuronovou sít si lze představit jako množinu vektorů, kde jeden perceptron odpovídá vektoru reálných čísel (vah vstupů + práh $v =(x_0,x_1...x_n,\Theta)$. V kontextu evolučních algoritmů se pro optimalizaci vektorů reálných čísel nejčastěji používají ES a DE, I z tohoto důvodu jsem zvolil zmíněné algoritmy jako zástupce pro optimalizaci chování heterogenní skupiny robotů. Oba zmíněné algoritmy důkladně popisuji v kapitole \ref{sec:DE} a \ref{sec:ES} a má implementace se od popisu v úvodu liší pouze v malý detailech. Do detailu si je lze prohlédnout v přiložené dokumentaci a kódu. 
\par
V rámci testovaní jsem vyzkoušel mnoho různých nastavení parametrů EA. V tabulce \ref{tab04:nastaveníEA} jsou uvedeny nakonec použité parametry, které dosahovali v experimentech největších úspěchů. \improvement{Popsat co jednotlivé parametry dělají??}Jedná se o tradičně používané parametry, osvědčené v řadě optimalizačních problémů. 
\begin{table}[h]\centering
	\begin{tabular}{l@{\hspace{1.5cm}}D{.}{,}{3.2}D{.}{,}{1.2}D{.}{,}{2.3}}
		\toprule
		 \textbf{differenciální evoluce}\\
		\midrule
		F:     & 0.8 \\
		CR:  & 0.5 \\
		\toprule
		\textbf{evoluční strategie}\\
		\midrule
		alpha & 0.05 \\
		sigma & 0.1\\
		\bottomrule
		\multicolumn{2}{l}{}
	\end{tabular}
	\caption{Nastavení parametrů u EA}
	\label{tab04:nastaveníEA}
\end{table}
\newpage
\section{Wood Scene experiment}
Vzorem Wood Scene scénáře byla těžba dřeva, představme si dřevorubce s motorovou pilou a silné dělníky nakládající zpracované stromy do transportérů a svážející materiál na společnou hromadu. Roboti odpovídají těmto lidským rolím, samozřejmě je jejich činnost značně zjednodušena. V obou případech je cílem maximalizovat počet zpracovaného dřeva na daném místě, což vyžaduje od obou druhů agentů spolupráci. \par
Robotické hejno se ve Wood Scene snaží natěžit a převést, co největší množství dřeva na místo označené rádiovým signálem. Rádiové senzory poskytují robotům sílu signálu a vysílaný kód. Pro místo určené na kupení dřeva je určen unikátní kód 2 a je umístěn doprostřed mapy. Žádný jiný rádiový vysílač vysílající signály s kódem 2 se na mapě nenachází.  
\par
Celé hejno čítá 9 robotů, jedná se o dva různé druhy, které se liší velikostí, rychlostí, senzory i efektory. Na začátku experimentu jsou náhodně rozmístěny do středu mapy na stejném místě jako se rozprostírá skladovací prostor. Dále jsou na mapě náhodně umístěny stromy. Robot v kontextu scénáře nazývaný Scout odpovídá \uv{dřevorubci} v teoretickém vzoru, pohybuje se rychle, má menší rozměry, umí nalezený strom zpracovat na dřevo pro komunikaci má přidělený unikátní kód 0. Oproti tomu robot \uv{dělník} se pohybuje pomaleji, je větší, neumí zpracovávat stromy, ale disponuje nakladačem (vykladačem) a kontejnerem na 5 objektů. 
 \par 
Souhrnně hejno musí strom nalézt, přepracovat na dřevo, poté naložit a odvézt do středu. Celý tento proces zahrnuje typické úkoly pro robotický swarm jako rozprostření, vyhýbání se překážkám, komunikaci mezi jednotlivými agenty, nalezení cesty apod. Z návrhu je zřejmé, že na procesu se musí podílet oba druhy robotů.
\par
V rámci bakalářské práce byla připraven program zajišťující jednoduchou vizualizaci chování robotů. Jeho vizuální výstup můžete vidět na obrázku \ref{obr04:WoodSceneRandomStart}. Na ní si vysvětlíme jednotlivé entity nacházející se na mapě. Mapa je ohraničena obdélníkovou hranicí a chová se jako zeď. Zelené kroužky znázorňují stromy, které ještě nebyly objeveny. Objevený strom změní barvu na žlutou. Modře označený prostor je určen pro uskladnění zpracovaného dřeva, roboti jej zaznamenají jako rádiový signál. Hnědá kolečka zastupují pokácené dřevo. V některých podúkolech se objevuje dřevo už při inicializaci, proto ještě neobjevené má tmavší barvu a objevené světlejší. Pro roboty je v tomto prostoru vysílán rádiový signál. Roboti jsou vyplněny červenou barvou, jejich senzory a efektory mají černou barvu. Pro každý rádiový signál je určena jedna unikátní barva s alfa kanálem. 
\par
Pro potvrzení, že scénář není triviálně řešitelný. Bylo vygenerováno tisíc náhodných chování. Hodnoceny byly dle fitness funkce podúkolu kooperace popsaný níže. Nejlepší z nich můžete vidět těsně po inicializaci mapy \ref{obr04:WoodSceneRandomStart}. Výsledek krátce před 10 000. iterací je zachycen v obrázku \ref{obr04:WoodSceneRandomEnd}. Největší světla zelená plocha je volný prostor, kde se mohou roboti pohybovat. Světle růžový kruh je právě rádiové vysílání, protože všichni roboti vysílají současně je barva celkem sytá. Tmavě fialový kruh označuje místo pro uskladnění, obvykle má modrý odstín, ale protože jej překrývají signály robotů zbarvil se do fialové. Na obrázku \ref{obr04:WoodSceneRandomEnd} vidíme, že se robotům povedlo jeden strom pokácet a dokonce uložit. Většina velkých robotů se dostala do kolize a menší roboti nedokázali objevit ani pětinu stromů.
\begin{figure}[p]\centering
	\includegraphics[width=\columnwidth]{../img/WoodMap/pictures/StartRandom.png}
	\caption{Příklad WoodScene mapy: start náhodného chování}
	\label{obr04:WoodSceneRandomStart}
\end{figure}
\par
\begin{figure}[p]\centering
	\includegraphics[width=\columnwidth]{../img/WoodMap/pictures/EndRandom.png}
	\caption{Příklad WoodScene mapy: po 9000 iteracích náhodného chování}
	\label{obr04:WoodSceneRandomEnd}
\end{figure}
\clearpage
\subsection{Roboti}
Devítičlenné hejno obsahuje 5 Scout robotů a 4 Worker roboty. U každého z robotů popíši jejich efektory a senzory. Pro každý druh robota je připravena jedna shodná neuronová sít. Jedinec odpovídá vektoru vah neuronové sítě. Pokud v rámci experimentu optimalizuji chování více druhů robotů, jedinec jsou dva vektory vah pro každý druh robota jedna neuronová síť. Proto fitness funkce hodnotí jejich výsledné snažení dohromady a evoluční operátory pracují nad celou dvojicí. Podívejme se na jednotlivé druhy podrobně.
\subsubsection{Scout robot}
Jedná se o robota, který má na starosti průzkum mapy a kácení nalezených stromů. Na zpracování dřeva používá efektor, nazývám jej refaktor, který má podobu úsečky a vyčnívá z čela robota. Pro zpracovaní musí refaktor kolidovat se stromem v mapě, poté je strom prohozen za entitu dřeva. Aby mohl komunikovat má přidělený rádiový signál s kódem 0, při jeho vysílání přidá na mapu signál jako kruh se středem odpovídajícím pozici robota. Jedná se o menšího robota, oproti Worker robotovi je rychlejší a jeho senzory mají větší dosah. Tabulka \ref{tab04:Scout} obsahuje základní charakteristiky, počty a dosahy jednotlivých senzorů a efektorů.
\par 
\begin{table}[h]\centering
\begin{tabular}{l@{\hspace{1.0cm}}D{.}{,}{2.2}D{.}{,}{2.2}D{.}{,}{2.3}}
	\toprule
	\textbf{Scout Robot} \\
	\midrule
        Tvar: & Kruh & \\
        Poloměr: & 2,5 \\
        Název: & WoodCutterM \\
        Velikost kontejneru: & 0\\
        \midrule
        \textbf{Efektory} \\
        \midrule
        Motor: & Dvou kolečkový \\
        Maximální rychlost: & 3 \\
        Kód rádiového signálu: & 0 \\
        Poloměr signálu: & 200\\
        Refaktor: & Strom \Rightarrow Dřevo \\
        Dosah refaktoru: & 10\\
        Počet paměťových slotů: & 10 \\
        Obsah slotu: & float\\
        \midrule 
        \textbf{Senzory} \\
        \midrule
        Počet line senzorů: &  3 \\
        Délka line senzorů: & 50\\
        Orientace line senzorů: & 0^\circ, 45^\circ, -45^\circ \\
        Poloměr type senzoru: & 50\\
        Poloměr rádiového přijímače: &  100 \\
        Počet touch senzorů: & 8 \\  
        Lokátor senzor\\ 
	\bottomrule
	\multicolumn{2}{l}{}
\end{tabular}
\caption{Wood Scene - Scout robot popis}
\label{tab04:Scout}
\end{table}
\clearpage
\subsubsection{Worker robot}
Worker robot se stará o manipulaci a následný transport objektů na mapě. Pohybuje se pomaleji než Scout a také je o něco rozměrnější. Picker, úsečkový efektor sloužící pro nakládání a vykládání objektů, funguje na podobném principu jako refaktor pro naložení musí kolidovat s objektem a pro vyložení s ním nesmí nic kolidovat. Ke komunikaci mu byl vyhrazen rádiový signál s kódem 1. Sebrané objekty ukládá do kontejneru, kam se vejde celkem 5 entit a vykládat umí pouze entitu na vrchu. Tabulka \ref{tab04:Worker} popisuje další podrobnosti.
\par 
\begin{table}[h]\centering
	\begin{tabular}{l@{\hspace{1.0cm}}D{.}{,}{2.2}D{.}{,}{2.2}D{.}{,}{2.3}}
			\toprule
			\textbf{Worker Robot} \\
			\midrule
                Tvar: & Kruh\\
                Poloměr: & 5\\
                Název: & WoodWorkerM \\
                Velikost kontejneru: & 5\\
                \midrule
                \textbf{Efektory} \\
                \midrule
                Motor: & Dvou kolečkový \\
                Maximální rychlost: & 2 \\
                Kód rádiového signálu: & 0\\
                Poloměr signálu: & 200\\
                Dosah pickeru: & 10\\
                Počet paměťových slotů: &10 \\
                Obsah slotu: & float\\
                \midrule 
                \textbf{Senzory} \\
                \midrule
                Počet line senzorů: &  3\\
                Délka line senzorů: & 30\\
                Orientace line senzorů: & 0^\circ, 45^\circ, -45^\circ\\
                Poloměr rádiového přijímače: & 100 \\
                Počet touch senzorů: & 8 \\  
                Lokátorový senzor\\ 
	\bottomrule
\multicolumn{2}{l}{}
\end{tabular}
\caption{Wood Scene - Worker robot popis }
\label{tab04:Worker}
\end{table}
\clearpage
\subsection{Vyhodnocování Fitness}
Fitness funkce pro ohodnocení WoodScene scénáře probíhá vždy až na konci simulace. I když se úspěšnost v podúkolech  vždy posuzuje jinak, celou fitness funkci lze shrnout do následujícího cílů. Roboti jsou odměňováni za: 
\begin{enumerate}
        \item \textit{nalezené stromy} - stromy o které zavadil line senzor 
        \item \textit{pokácené stromy} - stromy, které refaktor změnil 
        \item \textit{sebrané dřevo} - zpracované dřevo, které mají roboti uvnitř kontejnerů 
        \item \textit{uskladněné dřevo} - dřevo, které dovezli na vyznačené místo 
\end{enumerate}
Trestáni za:
\begin{enumerate}
	\item \textit{kolize} - počet pokusů o pohyb při kterém by došlo ke kolizi 
	\item \textit{sebrané entity mimo dřevo} - počet entit v kontejnerech, které nejsou zpracované dřevo 
\end{enumerate}

\subsection{Podúkoly} 
Rozdělil jsem hlavní cíl na následující podúkoly. Jejich obtížnost postupně roste a finální metaúkol už odpovídá řešenému problému. Pro ES a DE jsem použil stejné, abych bylo možné porovnat jejich fungování. 
\begin{enumerate}
        \item vygenerování robotů = Na začátku je vygenerováno chování robotů naprosto náhodně. Pro každého robota, je vygenerována náhodná jednovrstvá neuronová síť. 
        \item učení chůze = Pro oba roboty je velmi důležité, aby se pohybovali bez kolizí po celé mapě a objevovali, co největší prostor. Roboti jsou vyvíjeni odděleně a fitness se soustředí na počet kolizí (záporným ohodnocením) a na nalezené stromy (kladným ohodnocením).
        \item těžba stromů = Scout roboty, kteří se už obstojně po mapě pohybují, je třeba naučit kácet stromy. Proto dalším  cílem ve fitness funkci je počet pokácených stromů. Nicméně stále také na počet stromů nalezených. 
        \item převoz dřeva = Správně pohybující chceme naučit sbírat vytěžené dřevo. Fitness hodnotí počet sebraného dřeva, případně i uskladněné dřevo. Na těchto mapách jsou už na začátku připraveny pouze entity zpracovaného dřeva.
        \item kooperace = V posledním experimentu, se hodnotí pouze sebrané a uskladněné dřevo. A evolvují se oba druhy robotů současně. 
\end{enumerate}
U každého experimentu uvedu \unsure{nebylo by lepší slovo než myšlenku}myšlenku, tabulku s přesným nastavením a poté graf s průběhem fitness jednotlivých EA plus jejich vzájemné porovnání. ES v rámci mutačních operátorů vytváří několik zmutovaných jedinců a na základě jejich fitness utváří potomka. Vyhodnocování fitness je časově nejnáročnější výpočet, protože se musí probíhat na mapě celá simulace. Aby časy běhu DE a ES byly srovnatelné, odpovídá velikost populace u DE, velikosti populace krát počet mutovaných jedinců u ES. Při porovnání EA zobrazuji průměrné hodnoty fitness jedinců v rámci daného podúkolu. 
\newpage
	\subsubsection{Scout chůze - nastavení experimentu}
	Nejdříve jsem se zaměřil na schopnost pohybu jednotlivých robotů po mapě. Oddělil jsem oba druhy od sebe, protože díky rychlejšímu pohybu Scout robotů EA optimalizovalo pouze jejich pohyb. Roboti byli oceněni za nalezené stromy, tato fitness je nutila rozprostřít se po mapě. Následuje tabulka s nastavením fitness a EA.
	\par
	\begin{table}[h]\centering
		\begin{tabular}{l@{\hspace{1.5cm}}D{.}{,}{3.2}D{.}{,}{1.2}D{.}{,}{2.3}}
			\toprule
			\textbf{Nastavení mapy a EA}\\
			\midrule
			Roboti:     & Scout-5 \\
			Počet generací: & 1000\\
			Počet iterací map & 1000\\
			Velikost generace(DE) & 200\\
			Počet jedinců(ES) & 10\\
			Počet mutovaný potomků(ES)&20\\
			Elitismus(ES)& Ano\\
			Elitismus(DE)& Ne \\
			\bottomrule
			\multicolumn{2}{l}{}
		\end{tabular}
		\begin{tabular}{l@{\hspace{1.5cm}}D{.}{,}{3.2}D{.}{,}{1.2}D{.}{,}{2.3}}
			\toprule
			\textbf{Fitness funkce}\\
			\midrule
			Hodnota nalezeného stromu &  10 \\
			Ostatní hodnoty: & 0\\
			Počet stromů: & 300\\
			Počet už pokácených stromů & 100\\
			\bottomrule
			\multicolumn{2}{l}{}
		\end{tabular}
		\caption{Scout chůze - nastavení experimentu}\label{tab04:ScoutWalk}
	\end{table}
    Výsledky experimentu ilustrují grafy na další stránce. Jednotlivé průběhy fitness na obrázku \ref{obr04:WalkESvsDE} ukazují střední hodnotu fitness a její rozptyl v závislosti na generaci, jedinci jsou kvůli přehlednosti sloučeni po 10 generacích. V obou případech docházelo k největšímu růstu do 200 generace (v grafech 20). Oba EA lze označit jako úspěšné, protože vygenerovaná chování objevila více než 50\% stromů na mapě, v případě DE dokonce více dvě třetiny. U ES jsem nepoužíval elitismus, proto křivka více osciluje než je tomu u DE.
	\begin{figure}[t]\centering
		\includegraphics[width=\columnwidth]{../img/WoodMap/DEvsES/WCuttorWalkMem}
		\caption{ Scout chůze - porovnání průměrné fitness ES a DE}
		\label{obr04:WalkESvsDE}
	\end{figure}
	\clearpage
	
	\subsubsection{Worker chůze - nastavení experimentu}
	Worker chůze aplikuje podobný postup jako v předchozím experimentu na Worker roboty. Jen jsem nehodnotil počet nalezených stromů, ale do každé mapy jsem umístil už zpracované stromy. Roboti tedy byli oceněni za nalezení právě tohoto dřeva. Opět fitness funkce nutí roboty, co nejvíce se rozprostřít po mapě a navíc ještě vyhýbat se nepokáceným stromům.
	\par
	 	\begin{table}[h]\centering
		\begin{tabular}{l@{\hspace{1.5cm}}D{.}{,}{3.2}D{.}{,}{1.2}D{.}{,}{2.3}}
			\toprule
			\textbf{Nastavení mapy a EA}\\
			\midrule
			Roboti:     & Worker-4 \\
			Počet generací: & 1000\\
			Počet iterací map & 1000\\
			Velikost generace(DE) & 200\\
			Počet jedinců(ES) & 10\\
			Počet mutovaný potomků(ES)&20\\
			Elitismus(ES)& Ano\\
			Elitismus(DE)& Ne \\
			\bottomrule
			\multicolumn{2}{l}{}
		\end{tabular}
		\begin{tabular}{l@{\hspace{1.5cm}}D{.}{,}{3.2}D{.}{,}{1.2}D{.}{,}{2.3}}
			\toprule
			\textbf{Fitness funkce}\\
			\midrule
			Hodnota nalezeného pokáceného stromu &  20\\
			Ostatní hodnoty: & 0\\
			Počet stromů: & 0\\
			Počet už pokácených stromů & 400\\
			\bottomrule
			\multicolumn{2}{l}{}
		\end{tabular}
		\caption{Worker chůze - nastavení experimentu}
		\label{tab04:WorkerWalk}
	\end{table}
		DE dokázal už ve 200. generaci najít většinu zpracovaného dřevo na mapě, jak můžeme vidět na grafu DE v obrázku \ref{obr04:WWalkESvsDE}. ES neoptimalizovalo v tomto případě příliš rychle a uvízlo v lokalním optimu, jak ukazuje křivka ES od 200. generace. Nejlepší jedinec optimalizovaný pomocí DE byl schopen nalézt až 3x zpracovaného než ten pomocí ES. Nutno dodat, že náhodná pozice se entit na mapě se líší pro ES a DE. Zkoušel jsme, proto měnit seed u generátoru náhodných čísel a výsledky odpovídaly grafům v \ref{obr04:WWalkESvsDE}..
		\begin{figure}[t]\centering
		\includegraphics[width=\columnwidth]{../img/WoodMap/DEvsES/WorkerWalkMem}
		\caption{Worker chůze - porovnání průměrné fitness ES a DE}
		\label{obr04:WWalkESvsDE}
	\end{figure}
	\clearpage 
	
	
	
	\subsubsection{Scout kácení - nastavení experimentu}
	Dalším úkolem pro Scout robota bylo nalezené stromy pokácet. Použil jsem tedy optimalizované neuronové sítě z experimentu Scout chůze a tentokrát přidal do fitness pozitivní body za pokácené stromy. Refaktor je o mnoho kratší než line sensory, proto pro kácení musí robot ke stromu přijet blíže. Abych ještě více vylepšil pohyb po mapě, tak každá kolize byla potrestána negativním bodem do fitness. Přesné nastavení obsahuje následující tabulka.
	\begin{table}[h]\centering
		\begin{tabular}{l@{\hspace{1.5cm}}D{.}{,}{3.2}D{.}{,}{1.2}D{.}{,}{2.3}}
			\toprule
			\textbf{Nastavení mapy a EA}\\
			\midrule
			Roboti:     & Scout-5 \\
			Počet generací: & 1500\\
			Počet iterací map & 1000\\
			Velikost generace(DE) & 200\\
			Počet jedinců(ES) & 10\\
			Počet mutovaný potomků(ES)&20\\
			Elitismus(ES)& Ano\\
			Elitismus(DE)& Ne \\
			\bottomrule
			\multicolumn{2}{l}{}
		\end{tabular}
		\begin{tabular}{l@{\hspace{1.5cm}}D{.}{,}{3.2}D{.}{,}{1.2}D{.}{,}{2.3}}
			\toprule
			\textbf{Fitness funkce}\\
			\midrule
			Hodnota nalezeného stromu &  1000\\
			Hodnota pokáceného stromu & 10000\\
			Hodnota kolize & -1\\
			Ostatní hodnoty: & 0\\
			Počet stromů: & 400\\
			Počet už pokácených stromů & 0\\
			\bottomrule
			\multicolumn{2}{l}{}
		\end{tabular}
		\caption{Scout kácení - nastavení experimentu}
	\end{table}
	Dále můžete vidět grafy popisující průběh fitness u DE, ES a porovnání jejich průměrné fitness. Z \ref{obr04:CutESvsDE} můžeme vyčíst, že použité DE více cílí na exploataci a ES na exploraci. DE jsou díky tomu mnohem náchylnější k uvíznutí v lokálním optimu.  Fitness se v tomto případě skládá ze dvou složek, pro jedince bylo mnohem jednodušší stromy objevovat a náhodou nějaké pokácet. V prvních 650 generacích DE se tedy drží tento trend a pak se objeví jedinci, kteří cílí na kácení. Toto chování se rychle rozšířilo a fitness celé populace okolo 700. generace prudce vzrostla. Zatímco fitness v ES rostla postupně, ale nedosáhla tak vysoké úrovně jako DE. Nejlepší jedinci pokácí více než 60\% stromů. 
	\begin{figure}[t]\centering
		\includegraphics[width=\columnwidth]{../img/WoodMap/DEvsES/WCuttorCutMem}
		\caption{Scout kácení - porovnání průměrné fitness ES a DE}
		\label{obr04:CutESvsDE}
	\end{figure}
	\clearpage
	
	\subsubsection{Worker sbírání - nastavení experimentu}
	Worker v rámci zadání musí řešit více složitý úkol než Scout robot. Celý proces uložení se skládá nejdříve z nalezení dřeva, naložení, poté přesunu místa pro skladování a nakonec vyložení. Optimalizace fungovala lépe po rozdělení na část nakládání, kterou se zabývá tento experiment a na část vykládání, na kterou se podíváme později.  Ve fitness jsem se soustředil na správné entity v kontejneru a na jejich počet. Abych nezahodil dobré chování, které dokáže dřevo i ukládat, za uložení dřeva dostali roboti také pozitivní body. Níže můžete vidět tabulku s konkrétním nastavením. \par
	\begin{table}[h]\centering
		\begin{tabular}{l@{\hspace{1.5cm}}D{.}{,}{3.2}D{.}{,}{1.2}D{.}{,}{2.3}}
			\toprule
			\textbf{Nastavení mapy a EA}\\
			\midrule
			Roboti:     & Worker-4 \\
			Počet generací: & 2000\\
			Počet iterací map & 1000\\
			Velikost generace(DE) & 200\\
			Počet jedinců(ES) & 10\\
			Počet mutovaný potomků(ES)&20\\
			Elitismus(ES)& Ano\\
			Elitismus(DE)& Ne \\
			\bottomrule
			\multicolumn{2}{l}{ }
		\end{tabular}
		\par 
		\begin{tabular}{l@{\hspace{1.5cm}}D{.}{,}{3.2}D{.}{,}{1.2}D{.}{,}{2.3}}
			\toprule
			\textbf{Fitness funkce}\\
			\midrule
			Hodnota nalezeného pokáceného stromu &  100 \\
			Hodnota uloženého dřeva & 1010\\
			Hodnota dřeva v kontejneru & 1000\\
			Hodnota jiné entity v kontejneru & -100\\
			Hodnota kolize & -1\\
			Ostatní hodnoty: & 0\\
			Počet stromů: & 200\\
			Počet už pokácených stromů & 200\\
			\bottomrule
			\multicolumn{2}{l}{}
		\end{tabular}
		\caption{Worker sbírání - nastavení experimentu}
	\end{table}
	V grafech na \ref{obr04:PickupESvsDE} je vidět, že roboti dokázali maximálně naplnit kontejnery zpracovaným dřevem, v případě DE už po 250. generaci zvládala tento úkol celá populace. Chování optimalizované ES mají mnohem větší rozptyl díky vysoké exploraci, ale i její nejlepší jedinci zvládli maximální naplnění již kolem 250. generace. 
		   \begin{figure}[t]\centering       
		\includegraphics[width=\columnwidth]{../img/WoodMap/DEvsES/WorkerPickUpMem}
		\caption{ Worker sbírání - porovnání průměrné fitness ES a DE}
		\label{obr04:PickupESvsDE}
	\end{figure}
	\clearpage
	\subsubsection{Worker ukládání doprostřed  - nastavení experimentu}
	Ukládání zpracovaného dřeva byl poslední experiment, který plnili Worker roboti samostatně. Zde jsem se soustředil na celý úděl Worker robota a využil už optimalizovaných neuronových sítí na sbírání zpracovaného dřeva z předchozího experimentu. Nejvyšší odměnu roboti získávali za uskladnění dřeva a minoritní odměny za vhodné chování dle předchozích fitness. Vlastním nastavení odpovídá tabulce \ref{tab04:WorkerStore}. \par
	\begin{table}[h]\centering   
		\begin{tabular}{l@{\hspace{1.5cm}}D{.}{,}{3.2}D{.}{,}{1.2}D{.}{,}{2.3}}
			\toprule
			\textbf{Nastavení mapy a EA}\\
			\midrule
			Roboti:     & Worker-4 \\
			Počet generací: & 2000\\
			Počet iterací map & 2000\\
			Velikost generace(DE) & 200\\
			Počet jedinců(ES) & 10\\
			Počet mutovaný potomků(ES)&20\\
			Elitismus(ES)& Ano\\
			Elitismus(DE)& Ne \\
			\bottomrule
			\multicolumn{2}{l}{ }
		\end{tabular}
		\par 
		\begin{tabular}{l@{\hspace{1.5cm}}D{.}{,}{3.2}D{.}{,}{1.2}D{.}{,}{2.3}}
			\toprule
			\textbf{Fitness funkce}\\
			\midrule
			Hodnota nalezeného pokáceného stromu &  100 \\
			Hodnota uloženého dřeva & 1000\\
			Hodnota dřeva v kontejneru & 100\\
			Hodnota jiné entity v kontejneru & -100\\
			Hodnota kolize & -1\\
			Ostatní hodnoty: & 0\\
			Počet stromů: & 200\\
			Počet už pokácených stromů & 200\\
			\bottomrule
			\multicolumn{2}{l}{}
		\end{tabular}
		\caption{Worker ukládání doprostřed  - nastavení experimentu}
		\label{tab04:WorkerStore}
	\end{table}
   	Lokace skladovacího místa působila robotům velké obtíže, proto růst fitness byl mnohem mírnější než u předchozích experimentů. DE dokázala už optimalizované chování rychleji vylepšovat narozdíl od ES, která doplácí na velkou míru explorace viz. \ref{obr04:StockESvsDE}. Nejlepší jedinci jsou ovšem už schopni plnit obstojně hlavní úkol Worker robotů. Objevila se obtíž s efektivním urovnáním zpracovaného dřeva na skladovací místo. Pokoušel jsem se tento problém řešit, přidáním kladných bodů za menší vzdálenost mezi středem skladovacího místa a uloženým dřevem. Nesáhl jsem, však lepších výsledků než v tomto experimentu. 
	\begin{figure}[t]\centering
		\includegraphics[width=\columnwidth]{../img/WoodMap/DEvsES/WorkerStockMem}
		\caption{Worker ukládání doprostřed  - porovnání průměrné fitness ES a DE}
		\label{obr04:StockESvsDE}
	\end{figure}


	\clearpage
	\subsubsection{Kooperace  hlavní úkol  - nastavení experimentu}
	V rámci posledního úkolu jsem složil už optimalizovaná chování dohromady, abych vytvořil heterogenní hejno řešící problém Wood Scene scénáře. Spojoval jsem náhodné chování pro Scout robota s náhodným u Worker robota, aby se našla nejlepší jejich kombinace, tak bylo nutné navýšit počet generací na 4000. Fitness funkce(\ref{tab04:Coop}) byla v tomto případě průnik posledních experimentů Worker ukládání doprostřed a Scout kácení. 
	\begin{table}[h]\centering   
		\begin{tabular}{l@{\hspace{1.5cm}}D{.}{,}{3.2}D{.}{,}{1.2}D{.}{,}{2.3}}
		\toprule
		\textbf{Nastavení mapy a EA}\\
		\midrule
			Roboti: & Scout-5, Worker-4 \\
			Počet generací: & 4000\\
			Počet iterací map & 2000\\
			Velikost generace(DE) & 200\\
			Počet jedinců(ES) & 10\\
			Počet mutovaný potomků(ES)&20\\
			Elitismus(ES)& Ano\\
			Elitismus(DE)& Ne \\
			\bottomrule
			\multicolumn{2}{l}{}
		\end{tabular}
		\par 
		\begin{tabular}{l@{\hspace{1.5cm}}D{.}{,}{3.2}D{.}{,}{1.2}D{.}{,}{2.3}}
			\toprule
			\textbf{Fitness funkce}\\
			\midrule
			Hodnota nalezeného pokáceného stromu &  100 \\
			Hodnota uloženého dřeva & 1000\\
			Hodnota dřeva v kontejneru & 100\\
			Hodnota jiné entity v kontejneru & -100\\
			Hodnota kolize & -1\\
			Ostatní hodnoty: & 0\\
			Počet stromů: & 400\\
			Počet už pokácených stromů & 0\\
			\bottomrule
			\multicolumn{2}{l}{}
		\end{tabular}
			\caption{Kooperace  hlavní úkol  - nastavení experimentu}
			\label{tab04:Coop}
	\end{table}

	Na \ref{obr04:CoopESvsDE} lze pozorovat, že pro ES bylo velmi obtížné skládat už úspěšné chování do lepších celků a velmi osciluje. ES uškodilo agresivní prohledávání prostoru a DE díky své architektuře vytěžilo z optimalizovaných chování maximum. Nejlepšímu jedinci se budeme věnovat podrobněji v další kapitole. 
		\clearpage
		\begin{figure}[t]\centering
		\includegraphics[width=\columnwidth]{../img/WoodMap/DEvsES/WoodCoopMem}
		\caption{Kooperace  hlavní úkol  - porovnání průměrné fitness ES a DE}
		\label{obr04:CoopESvsDE}
	\end{figure}
	\clearpage
	
	\subsection{Výsledky Experimentu}
	Výsledkem posloupnosti všech podúkolů vzniklo poměrně velmi komplexní chování. Finální neuronové sítě ať u Worker robotů, či Scout robotů se zvládají vyhýbat překážkám. Scout roboti kácí stromy, které naleznou. Worker roboti nakládají zpracované dřevo, pokud na něj narazí, když zachytí signál úložiště, tak vyloží aktuální náklad. Některá chování byla také schopna předejít zaseknutí o nějaký shluk objektů, pokud byl  jejich pohyb vpřed neúspěšný, tak po několika pokusech roboti vycouvali a vydali se cestou okolo kritického místa. U většiny se také objevilo použití rádiových signálů jako prostředku pro největší možné rozptýlení po mapě, jakmile zachytí cizí signál vydají se opačným směrem. Průběh fitness jednotlivých podúkolů je zachycena na předchozích grafech \ref{obr04:Walk} až \ref{obr04:StockES}. V tabulce \ref{tab04:WoodStat} můžete vyčíst průměrné počty nalezených stromů, uskladněného materiálu, apod ze 100 simulací mapy. 
	
	Ač se jedná o nejlepší dosažené chování objevují se nějaké nedostatky. Worker robot se občas dostane do pozice, ze které není schopen vyjet, jedná se především o kolize s vícero entitami. Tento problém by mohlo vyřešit použití vícevrstvých sítí či evolučního algoritmu evolvujícího i architekturu sítě. Skládání zpracovaného materiálu po obvodu skladiště není efektivní způsob, jak do něj naskládat maximální množství dřeva. V tomto případě jsem se snažil vylepšit tento nedostatek promítnutím vzdálenosti dřeva od středu skladiště do celkové fitness, ovšem bez znatelného zlepšení v chování. Nejspíše by bylo třeba použít rádiový senzor poskytující více informací o směru k zachycenému signálu. 
	
		\begin{table}[h]\centering   
		\begin{tabular}{l@{\hspace{1.5cm}}D{.}{,}{3.2}D{.}{,}{1.2}D{.}{,}{2.3}}
			\toprule
			& \mc{} & \mc{}\\
		\textbf{Inicializační nastavení:}  \\
			\midrule
			Výška & 800\\ 
			Šířka & 1200\\
			Počet iterací & 10000\\
			Počet stromů & 400\\
			Počet Scout robotů & 5\\
			Počet Worker robotů & 4\\
			\bottomrule
			\multicolumn{2}{l}{}
		\end{tabular}
		\caption{WoodScene - nastavení mapy pro testovací experiment}
	\end{table}
	\begin{table}[h]\centering   
		\begin{tabular}{l@{\hspace{1.5cm}}D{.}{,}{3.2}D{.}{,}{1.2}D{.}{,}{2.3}}
			\toprule
			& \mc{} & \mc{}\\
			\textbf{Výsledky} \\
			\bottomrule
			Zpracované dřevo zanechané v mapě & 225.22\\
			Stromy v mapě & 156.22\\
			Z toho nalezené & 52.84\\
			Dřevo v kontejnerech & 18.48\\
			Uskladněné dřevo & 17.56\\
			\multicolumn{2}{l}{}
		\end{tabular}
		\caption{WoodScene - výsledky simulace nejlepšího jedince, průměr ze 100 simulací testovacího experimentu}
		\label{tab04:WoodStat}
	\end{table}
	\newpage 
	\begin{figure}[p]\centering
		\includegraphics[width=\columnwidth]{../img/WoodMap/pictures/end.png}
		\caption{Nejlepší jedinec - 10000 iterací simulace}
		\label{obr04:bestEnd}
	\end{figure}
	\begin{figure}[p]\centering
		\includegraphics[width=\columnwidth]{../img/WoodMap/pictures/EndRandom.png}
		\caption{Náhodný jedinec - 10000 iterací simulace}
		\label{obr04:randomEnd}
	\end{figure}
	\clearpage
\section{Mineral Scene}
Tento scénář si bere jako inspiraci strategické hry např. Starcraft \citep*{starcraft} a hypotetické přežití robotů na cizí planetě, kde si budou muset obstarat vlastní nerostné suroviny pro běh. Opět je klíčová spolupráce mezi roboty, kdy roboti, kteří mají různé cíle. Jejich společným cílem je maximalizovat množství vyrobeného paliva.
\par 
V Mineral Scene scénáři se palivo se vyrábí z minerálů, které jsou náhodně rozmístěny po mapě. Transformaci minerálu na palivo dokáže pouze největší robot \textit{Refactor}. V mapě se dále nacházejí překážky a volné palivo, které mohou roboti ihned po sebrání použít. V rámci hlavního cíle roboti musí nejdříve nalézt minerál, Refactor jej přeměnit a pak případně předat ostatním robotům. \unsure{Je to hodně komplexní asi to zjednoduším}
\par
V Mineral Scene figurují dohromady tři rozliční roboti, všichni potřebují pro pohyb odlišné množství paliva. Nejmenší robot \textit{Mineral Scout} disponuje pouze senzory k exploraci prostředí a rádiovým vysílačem pro komunikaci se skupinou, opět má unikátní kód  jako v předchozím scénáři. Robot střední velikosti \textit{Mineral Worker} se pohybuje o něco pomaleji než Mineral Scout, ale umí přesouvat objekty i více najednou. Robot pro přeměnu minerálu (suroviny na výrobu paliva) označen v simulaci jako Mineral Refactor se přemisťuje nejpomaleji, má možnost přeměnit minerál na palivo. 
\par
Jedná se o složitější cíl než v předchozím scénáři. V mapě se vyskytují navíc překážky a celkový počet entit na mapě je vyšší. Zatímco ve Wood Scene je místo pevně místo přesunu surovin pěkně dané, zde se musí transportní roboti hledat Refaktor roboty.
\par 
Více detailů u obrázků \ref{obr04:MineralSceneRandomStart}.
\newpage
Na obrázku \ref{obr04:MineralSceneRandomStart} je vyobrazena vizualizace daného scénáře. Roboti jsou vybarveni červenou barvou a jednotlivé druhý od sebe lze snadno rozeznat podle velikosti (nejmenší Scout robot, poté Worker robot a největší Refaktor robot), Zelené kroužky představují minerály, pokud je minerál objevený hejnem obarví se žlutě. Šedivou barvou jsou vyvedeny překážky a palivo černou.

\begin{figure}[h]\centering
	\includegraphics[width=\columnwidth]{../img/MineralMap/MineralRandom.png}
	\caption{Příklad Mineral Scene mapy: konfigurace po startu s  náhodným chováním}
	\label{obr04:MineralSceneRandomStart}
\end{figure}
\clearpage 
\subsection{Roboti}
V Mineral Scene scénáři se vyskytuje 12 robotů. V hejnu se vyskytují 3 různé druhy robotů: Scout robot, Worker robot, Refaktor robot. V implementaci jsou odlišeni od předchozího experimentu prefixem Mineral. Jejich ovládání stejně jako v předchozím experimentu probíhá pomocí neuronových sítí. Jedinec má opět podobu vektoru vah jednotlivých neuronových sítí a hodnoceno je celé hejno. Každý robot má nádrž na palivo a každá iterace mapy ho stojí jednu jednotka paliva. Nyní popíšeme jednotlivé roboty.
\subsubsection{Scout robot}
Jak název napovídá Scout robot prozkoumává mapu. Oproti Wood Scene nemá žádnou další funkci. Ze všech robotů se pohybuje nejrychleji a díky malé velikosti dokáže projíždět i užšími prostory. Informace o prostředí získává pomocí úsečkových senzorů a velkého kruhového senzoru, který poskytuje počet jednotlivých druhů entit v celém okruhu. Pro komunikaci se zbytkem používá rádiový signál s kódem 0. 
\par  

\begin{table}[h]\centering
	\begin{tabular}{l@{\hspace{1.0cm}}D{.}{,}{2.2}D{.}{,}{2.2}D{.}{,}{2.3}}
		\toprule
		\textbf{Scout Robot} \\
		\midrule
		Tvar: & Kruh\\
		Poloměr: & 2,5\\
		Namespace: & MineralRobots\\
		Název: & ScoutRobotMem \\
		Velikost kontejneru: & 0\\
		\midrule
		\textbf{Efektory} \\
		\midrule
		Motor: & Dvou kolečkový \\
		Maximální rychlost: & 3 \\
		Kód rádiového signálu: & 0\\
		Poloměr signálu: & 200\\
		Počet paměťových slotů: &10 \\
		Obsah slotu: & float\\
		\midrule
		\textbf{Senzory} \\
		\midrule
		Počet line senzorů: &  3\\
		Délka line senzorů: & 70\\
		Orientace l. senzorů: & 0^\circ,\pm 45^\circ\\
		Počet fuel senzorů: &  3\\
		Délka fuel senzorů: & 70\\
		Orientace f. senzorů: & 0^\circ, \pm 45^\circ\\
		Poloměr rádiového přijímače: & 100 \\
		Počet touch senzorů: & 3 \\  
		Lokátorový senzor\\ 
		\bottomrule
		\multicolumn{2}{l}{}
	\end{tabular}
	\caption{Mineral Scene - Scout robot specifikace }
	\label{tab04:MineralScout}
\end{table}
\clearpage
\subsubsection{Worker Robot}
Mineral Worker roboti zastupují úlohu transportu minerálů. Pohybují se rychleji než Refaktor roboti, ale zase pomaleji než Scout roboti. Uloží do svého kontejneru až 5 minerálů. Pro komunikaci využívají rádiového signálu s kódem 1. Má k dispozici Picker pro nákládání a vykládání entit z kontejneru.
\begin{table}[h]\centering
	\begin{tabular}{l@{\hspace{1.0cm}}D{.}{,}{2.2}D{.}{,}{2.2}D{.}{,}{2.3}}
		\toprule
		\textbf{Worker Robot} \\
		\midrule
		Tvar: & Kruh\\
		Poloměr: & 5 \\
		Namespace: & MineralRobots\\
		Název: & WorkerRobotMem \\
		Velikost kontejneru: & 5\\
		\midrule
		\textbf{Efektory} \\
		\midrule
		Motor: & Dvou kolečkový \\
		Maximální rychlost: & 1,5 \\
		Kód rádiového signálu: & 1\\
		Poloměr signálu: & 200\\
		Dosah pickeru: & 10\\
		Počet paměťových slotů: &10 \\
		Obsah slotu: & float\\
		\midrule 
		\textbf{Senzory} \\
		\midrule
		Počet line senzorů: &  3\\
		Délka line senzorů: & 50\\
		Orientace l. senzorů: & 0^\circ, \pm45^\circ\\
		Počet fuel senzorů: &  3\\
		Délka fuel senzorů: & 50\\
		Orientace f. senzorů: & 0^\circ, \pm45^\circ\\
		Poloměr rádiového přijímače: & 100 \\
		Počet touch senzorů: & 3 \\  
		Lokátorový senzor\\ 
		\bottomrule
		\multicolumn{2}{l}{}
	\end{tabular}
	\caption{Mineral Scene - Worker robot specifikace }
	\label{tab04:MineralWorker}
\end{table}
\clearpage
\subsubsection{Refaktor robot}
Nenahraditelnou roli zastává Mineral Scene, dokáže totiž měnit minerál na jednotku paliva. Pro přeměnu musí být minerál připraven na vrcholu kontejneru a po procesu přeměny se místo minerálu objeví palivo. Refaktor robot je však oproti ostatním robotů značně neohrabaný, zvláště kvůli jeho velikosti. Poloměr Refaktor robota odpovídá dvěma Worker robotům (resp. čtyřem Scout robotům). Pohybuje se z nich také nejpomaleji. Jeho rádiové signály nesou kód 2. Nakladače (vykladače) má do všech čtyrech světových směrů. 
\par  
\begin{table}[h]\centering
	\begin{tabular}{l@{\hspace{1.0cm}}D{.}{,}{2.2}D{.}{,}{2.2}D{.}{,}{2.3}}
		\toprule
		\textbf{Refaktor Robot} \\
		\midrule
		Tvar: & Kruh\\
		Poloměr: & 10 \\
		Namespace: & MineralRobots\\
		Název: & RefactorRobotMem \\
		Velikost kontejneru: & 5\\
		\midrule
		\textbf{Efektory} \\
		\midrule
		Motor: & Dvou kolečkový \\
		Maximální rychlost: & 1,5 \\
		Kód rádiového signálu: & 2\\
		Poloměr signálu: & 200\\
		Počet pickerů & 4\\
		Orientace pickerů & 0^\circ, 90^\circ, 180^\circ,270^\circ\\ 
		Dosah pickerů: & 20\\
		Počet paměťových slotů: &10 \\
		Obsah slotu: & float\\
		Refaktor: & Minerál \Rightarrow Palivo \\
		Dosah refaktoru:  & kontejner \\
		Kapacita refaktoru: & 1\\ 
		\midrule 
		\textbf{Senzory} \\
		\midrule
		Počet line senzorů: &  3\\
		Délka line senzorů: & 70\\
		Orientace l. senzorů: & 0^\circ, \pm 35^\circ\\
		Počet fuel senzorů: &  3\\
		Délka fuel senzorů: & 70\\
		Orientace f. senzorů: & 0^\circ, \pm 35^\circ\\
		Poloměr rádiového přijímače: & 100 \\
		Počet touch senzorů: & 3 \\  
		Lokátorový senzor\\ 
		\bottomrule
		\multicolumn{2}{l}{}
	\end{tabular}
	\caption{Mineral Scene - Refaktor robot specifikace }
	\label{tab04:MineralRefactor}
\end{table}
\clearpage
\subsection{Vyhodnocování Fitness}
Fitness funkce pro Mineral Scene scénář jsem, po dobrých zkušenostech z předchozího experimentu, použil vážený součet následujících charakteristik mapy po konci simulace. Tento součet je specifický pro každý podúkol zvlášť. Kvůli komplexnosti toho úkolu a velikosti hejna v jednotlivých podúkolech nevystupují roboti vždy v plném počtu, ale nejdříve se optimalizuje chování pro pár jedinců od každého druhu. I tak bylo potřeba zmenšit velikost populace pro rozumnou dobu času běhu. Pozitivní hodnocení roboti získávali za: 
\begin{itemize}
	\item \textit{objevené minerály} - minerály o které zavadil line senzor
	\item \textit{uložené minerály} - minerály nacházející se v kontejnerech robotů 
	\item \textit{přeměněné palivo} - palivo nacházející se na mapě či v kontejnerech robotů
	\item \textit{palivo v nádržích} - palivo uvnitř nádrží jednotlivých robotů
	\item \textit{odložené minerály} - minerály na pomocném prostoru označeným rádiovým signálem
\end{itemize}
A negativní pouze za: 
\begin{itemize}
	\item \textit{kolize} - počet pokusů o pohyb který by vedl ke kolizi
\end{itemize}
\subsection{Podúkoly}
Podle výsledků předchozího experimentu jsem se tentokrát výhradně soustředil na DE, které vycházelo v celkovém důsledku jako lepší.
\par 
Opět jsem vygeneroval první neuronové sítě náhodně i další kroky jsou podobné jako v předchozím případě, nejdříve jsem navrhl fitness funkce pro učení chůze, vyhýbání, sbírání objektů a jejich skládání. Dále jsem však nebyla zřejmá posloupnost jednotlivých úkonů, proto fitness funkce odpovídá už výslednému cíli, množství vytvořeného paliva a sebraným minerálů.
\par 
Od následující stránky se budu soustředit na jednotlivé podúkoly. U každého metaúkolu bude uveden krátký popis cíle jednotlivého experimentu, obrázek s grafy průběhem fitness, stručné shrnutí chování a grafů s průběhy. Grafy budou vždy stejného formátu,  y osa zobrazuje hodnotu fitness, x osa odpovídá číslu generace, pro lepší čitelnost jsou generace sdruženy po 10. 
\clearpage

\subsubsection{Scout chůze - nastavení experimentu}
Stejně jako u předchozího scénáře, jsem roboty nejdříve učil pohybu. Opět jsem roboty oddělil, aby optimalizace neupřednostňovala ty rychlejší. Náhodně vygenerované neuronové sítě pro řízení Scout robotů. Poté jsou optimalizovány pomocí fitness, která oceňuje chování podle počtu objevených minerálů.
\begin{table}[h]\centering   
	\begin{tabular}{l@{\hspace{1.5cm}}D{.}{,}{3.2}D{.}{,}{1.2}D{.}{,}{2.3}}
		\toprule
		\textbf{Nastavení mapy a EA}\\
		\midrule
		Roboti a jejich počet: & Scout-5 \\
		Počet generací: & 1000\\
		Počet iterací map & 1000\\
		Velikost generace(DE) & 100\\
		\bottomrule
		\multicolumn{2}{l}{}
	\end{tabular}
	\par 
	\begin{tabular}{l@{\hspace{1.5cm}}D{.}{,}{3.2}D{.}{,}{1.2}D{.}{,}{2.3}}
		\toprule
		\textbf{Fitness funkce}\\
		\midrule
		Hodnota nalezeného minerálu &  100 \\
		Ostatní hodnoty: & 0\\
		\toprule
		\textbf{Objekty na mapě}\\
		\midrule
		Počet minerálů: & 400\\
		Počet překážek & 100\\
		Počet paliva & 100\\
		\bottomrule
		\multicolumn{2}{l}{}
	\end{tabular}
	\caption{Scout chůze - nastavení experimentu}
	\label{tab04:MineralScoutWalk}
\end{table}
Z grafu \ref{obr04:MineralScoutWalk} lze vyčíst, že ani větší množství entit v mapě nevadilo pro optimalizaci slušného chování pro vyhýbání se překážkám a prohledávání mapy. Nejlepší chování dokáže odhalit okolo 75\% minerálů na mapě.
\clearpage
\begin{figure}[t]\centering
	\includegraphics[width=\columnwidth]{../img/MineralMap/MineralScoutWalk}
	\caption{Mineral Scout chůze -  průběh fitness DE}
	\label{obr04:MineralScoutWalk}
\end{figure}
Modrá křivka odpovídá průměrné hodnotě fitness, světle modře je pak vybarvena oblast daná průměrem $\pm$ směrodatná odchylka fitness. Generace (osa y) jsou sdruženy po 10. 
\clearpage

\subsubsection{Worker chůze - nastavení experimentu}
I pro Worker robota bylo potřeba optimalizovat chůzi v mapě. Stejně jako u Scout robotů byli Worker roboti odměňovány za nalezené Minerály. 
\par
\begin{table}[h]\centering   
	\begin{tabular}{l@{\hspace{1.5cm}}D{.}{,}{3.2}D{.}{,}{1.2}D{.}{,}{2.3}}
		\toprule
		\textbf{Nastavení mapy a EA}\\
		\midrule
		Roboti a jejich počet: & Worker-4 \\
		Počet generací: & 1000\\
		Počet iterací map & 1000\\
		Velikost generace(DE) & 100\\
		\bottomrule
		\multicolumn{2}{l}{}
	\end{tabular}
	\par 
	\begin{tabular}{l@{\hspace{1.5cm}}D{.}{,}{3.2}D{.}{,}{1.2}D{.}{,}{2.3}}
		\toprule
		\textbf{Fitness funkce}\\
		\midrule
		Hodnota nalezeného minerálu &  100 \\
		Ostatní hodnoty: & 0\\
		\toprule
		\textbf{Objekty na mapě}\\
		\midrule
		Počet minerálů: & 300\\
		Počet překážek & 100\\
		Počet paliva & 100\\
		\bottomrule
		\multicolumn{2}{l}{}
	\end{tabular}
	\caption{Mineral Worker chůze - nastavení experimentu}
	\label{tab04:MineralWorkerWalk}
\end{table}
Nejlepší chování u Worker robota dokáže objevit více než čtvrtinu minerálů, jak ukazuje obrázek \ref{obr04:MineralWorkerWalk} Což je také uspokojivý výsledek, přihlédneme-li k tomu, že Scout robot má poloviční rychlost a velikost než Worker robota. Navíc díky delším senzorům objevuje Scout robot objekty na větší vzdálenosti. 
\clearpage
\begin{figure}[t]\centering
	\includegraphics[width=\columnwidth]{../img/MineralMap/MineralWorkerWalk}
	\caption{Mineral Worker chůze -  průběh fitness DE}
	\label{obr04:MineralWorkerWalk}
\end{figure}
Modrá křivka odpovídá průměrné hodnotě fitness, světle modře je pak vybarvena oblast daná průměrem $\pm$ směrodatná odchylka fitness. Generace (osa y) jsou sdruženy po 10. 
\clearpage

\subsubsection{Worker sbírání - nastavení experimentu}
Stejně jako ve Wood Scene bylo klíčové sbírání už zpracováného dřeva, tak v Mineral Scene jsem cílil na maximální počet sebraného paliva. Na rozdíl od Wood Scene však není zcela jasné, kam ukládat sebrané minerály. Aby v budoucích podúkolech umisťovali roboti, co nejblíže k Refaktor robotům, přidal jsem proto prozatímně do středu mapy pomocný rádiový signál se stejným kódem jako mají Refaktor roboti. 
\par
\begin{table}[h]\centering   
	\begin{tabular}{l@{\hspace{1.5cm}}D{.}{,}{3.2}D{.}{,}{1.2}D{.}{,}{2.3}}
		\toprule
		\textbf{Nastavení mapy a EA}\\
		\midrule
		Roboti a jejich počet: & Worker-2\\
		Počet generací: & 3000\\
		Počet iterací map & 1000\\
		Velikost generace(DE) & 100\\
		\bottomrule
		\multicolumn{2}{l}{}
	\end{tabular}
	\par 
	\begin{tabular}{l@{\hspace{1.5cm}}D{.}{,}{3.2}D{.}{,}{1.2}D{.}{,}{2.3}}
		\toprule
		\textbf{Fitness funkce}\\
		\midrule
		Pomocný rádiový signál: & Ano\\
		Hodnota nalezeného minerálu &  1\\
		Hodnota minerálu v pomoc. signálu & 1010\\ 
		Hodnota uložených minerálů & 1000\\
		Ostatní hodnoty: & 0\\
		\toprule
		\textbf{Objekty na mapě}\\
		\midrule
		Počet minerálů: & 400\\
		Počet překážek & 100\\
		Počet paliva & 100\\
		\bottomrule
		\multicolumn{2}{l}{}
	\end{tabular}
	\caption{Mineral Worker sbírání - nastavení experimentu}
	\label{tab04:MineralWorkerPickUp}
\end{table}
Roboti v prvních 500 generacích zaplní své kontejnery minerály, což může pozorovat na  grafu \ref{obr04:MineralWorkerPickUp}. Dále dokonce dováželi roboti do středu několik minerálů. 
\clearpage
\begin{figure}[t]\centering
	\includegraphics[width=\columnwidth]{../img/MineralMap/MineralWorkerPickup}
	\caption{Mineral Worker sbírání - průběh fitness}
	\label{obr04:MineralWorkerPickUp}
\end{figure}
Modrá křivka odpovídá průměrné hodnotě fitness, světle modře je pak vybarvena oblast daná průměrem $\pm$ směrodatná odchylka fitness. Generace (osa y) jsou sdruženy po 10. 
\clearpage

\subsubsection{Worker skládání - nastavení experimentu}
V dalším metaúkolu jsem se soustředil výhradně na skládání minerálů do středu na místo označeném kódem 2. Konkrétní nastavení experimentu popisuje jako tabulka \ref{tab04:MineralWorkerStore}. Oproti předchozímu scénáři optimalizuji Workery v plném počtu, aby se optimalizovala komunikace a rozptylování Worker robotů. 
\par  
\begin{table}[h]\centering   
	\begin{tabular}{l@{\hspace{1.5cm}}D{.}{,}{3.2}D{.}{,}{1.2}D{.}{,}{2.3}}
		\toprule
		\textbf{Nastavení mapy a EA}\\
		\midrule
		Roboti a jejich počet: & Worker-4\\
		Počet generací: & 6000\\
		Počet iterací map & 1000\\
		Velikost generace(DE) & 100\\
		\bottomrule
		\multicolumn{2}{l}{}
	\end{tabular}
	\par 
	\begin{tabular}{l@{\hspace{1.5cm}}D{.}{,}{3.2}D{.}{,}{1.2}D{.}{,}{2.3}}
		\toprule
		\textbf{Fitness funkce}\\
		\midrule
		Pomocný rádiový signál: & Ano\\
		Hodnota nalezeného minerálu &  1\\
		Hodnota minerálu v pomoc. signálu & 1000\\ 
		Hodnota uložených minerálů & 10\\
		Ostatní hodnoty: & 0\\
		\toprule
		\textbf{Objekty na mapě}\\
		\midrule
		Počet minerálů: & 400\\
		Počet překážek & 100\\
		Počet paliva & 100\\
		\bottomrule
		\multicolumn{2}{l}{}
	\end{tabular}
	\caption{Mineral Worker skládání - nastavení experimentu}
	\label{tab04:MineralWorkerStore}
\end{table}
Výsledek tohoto podúkolu odpovídá Wood Scene - sbírání, roboti dokázali do středu převést přibližně 10 minerálů, opět měli obtíže s uspořádáním. Vizuální výsledek potvrzuje graf \ref{obr04:MineralWorkerStore}.
\clearpage
\begin{figure}[t]\centering
	\includegraphics[width=\columnwidth]{../img/MineralMap/MineralWorkerPickup}
	\caption{Mineral Worker skládání -  průběh fitness DE}
	\label{obr04:MineralWorkerStore}
\end{figure}
Modrá křivka odpovídá průměrné hodnotě fitness, světle modře je pak vybarvena oblast daná průměrem $\pm$ směrodatná odchylka fitness. Generace (osa y) jsou sdruženy po 10. 
\clearpage

\subsubsection{Refaktor chůze - nastavení experimentu}
Poslední metaúkol zabývající chůzí byl pro Refaktor robota i v jeho případě byl hodnocen podle počtu nalezených minerálů jako u ostatních robotů. Tabulka \ref{obr04:MineralRefaktorWalk} ukazuje přesné nastavení metaúkolu.
\par
\begin{table}[h]\centering   
	\begin{tabular}{l@{\hspace{1.5cm}}D{.}{,}{3.2}D{.}{,}{1.2}D{.}{,}{2.3}}
		\toprule
		\textbf{Nastavení mapy a EA}\\
		\midrule
		Roboti a jejich počet: & Refaktor-3\\
		Počet generací: & 1000\\
		Počet iterací map & 1500\\
		Velikost generace(DE) & 100\\
		\bottomrule
		\multicolumn{2}{l}{}
	\end{tabular}
	\par 
	\begin{tabular}{l@{\hspace{1.5cm}}D{.}{,}{3.2}D{.}{,}{1.2}D{.}{,}{2.3}}
		\toprule
		\textbf{Fitness funkce}\\
		\midrule
		Hodnota nalezeného minerálu &  1\\
		Ostatní hodnoty: & 0\\
		\toprule
		\textbf{Objekty na mapě}\\
		\midrule
		Počet minerálů: & 500\\
		Počet překážek & 100\\
		Počet paliva & 100\\
		\bottomrule
		\multicolumn{2}{l}{}
	\end{tabular}
	\caption{Mineral Refaktor chůze - nastavení experimentu}
	\label{tab04:MineralRefaktorWalk}
\end{table}
Už v 200 generaci dokážou Refaktor robot odhalí jednu pětinu minerálů, jak je vidět v průběhu grafu na \ref{obr04:MineralRefaktorWalk}. Ač podle fitness se zdá, že roboti prohledávají mapu a vyhýbají se překážkám. Po krátkém zkoumání jejich chování jsem zjistil, že používají nakladače a přehazují překážky za sebe.
\clearpage
\begin{figure}[h]\centering
	\includegraphics[width=\columnwidth]{../img/MineralMap/MineralRefaktorWalk}
	\caption{Mineral Refaktor chůze - průběh fitness u DE}
	\label{obr04:MineralRefaktorWalk}
\end{figure}
Modrá křivka odpovídá průměrné hodnotě fitness, světle modře je pak vybarvena oblast daná průměrem $\pm$ směrodatná odchylka fitness. Generace (osa y) jsou sdruženy po 10. 
\clearpage

\subsubsection{Refaktor Worker kooperace - nastavení experimentu}
Jako první kooperativní metaúkol jsem zvolil spolupráci Refaktor robota s Workera robota. Zamýšlel jsem, že Worker roboti seberou minerály a budou je přibližovat k Refaktor robotovi, který je bude přetvářet na palivo. Tomuto účelu jsem přizpůsobil ohodnocení fitness, jedinci jsou nejvíce oceněni za přeměněné palivo vložené do mapy či už v nádrži. Další podrobnosti v tabulce \ref{obr04:MineralRefactorWorkerCoop} 
\par
\begin{table}[h]\centering   
	\begin{tabular}{l@{\hspace{1.5cm}}D{.}{,}{3.2}D{.}{,}{1.2}D{.}{,}{2.3}}
		\toprule
		\textbf{Nastavení mapy a EA}\\
		\midrule
		Roboti: & Refaktor, Worker\\
		Počty robotů: & R-1, W-2 \\
		Počet generací: & 2000\\
		Počet iterací map & 1500\\
		Velikost generace(DE) & 100\\
		\bottomrule
		\multicolumn{2}{l}{}
	\end{tabular}
	\par 
	\begin{tabular}{l@{\hspace{1.5cm}}D{.}{,}{3.2}D{.}{,}{1.2}D{.}{,}{2.3}}
		\toprule
		\textbf{Fitness funkce}\\
		\midrule
		Hodnota uložených minerálů & 1\\
		Hodnota přeměného paliva & 1000\\ 
		Hodnota paliva v nádržích & 1000\\
		Ostatní hodnoty: & 0\\
		\toprule
		\textbf{Objekty na mapě}\\
		\midrule
		Počet minerálů: & 400\\
		Počet překážek & 100\\
		Počet paliva & 0\\
		\bottomrule
		\multicolumn{2}{l}{}
	\end{tabular}
	\caption{Mineral Refaktor Worker kooperace - nastavení experimentu}
	\label{tab04:MineralRefactorWorkerCoop}
\end{table}
Ač se podle grafu \ref{obr04:MineralRefactorWorkerCoop}, že optimalizaci obou robotů proběhla úspěšně. Ale po vizuální kontrole jsem zjistil, že se zlepšoval pouze Refaktor robot. EA však v rámci nejlepších chování nezapojila Worker robota.
\clearpage
\begin{figure}[h]\centering
	\includegraphics[width=\columnwidth]{../img/MineralMap/MineralWorkerRefaktorCoop}
	\caption{Mineral Refaktor Worker kooperace -  průběh fitness DE}
	\label{obr04:MineralRefactorWorkerCoop}
\end{figure}
Modrá křivka odpovídá průměrné hodnotě fitness, světle modře je pak vybarvena oblast daná průměrem $\pm$ směrodatná odchylka fitness. Generace (osa y) jsou sdruženy po 10. 
\clearpage
\subsubsection{ Hlavní úkol kooperace - nastavení experimentu}
V rámci finálního podúkolu byla fitness nastavena, aby odpovídala hlavnímu cíli scénáře.  Oceňuje roboty pouze za přetvořené palivo a palivo v nádržích. V tomto podúkolu už figurují všechny druhy robotů. Další podrobnosti jsou zaneseny v tabulce \ref{tab04:MineralFullCoop}. 
\par
\begin{table}[h]\centering   
	\begin{tabular}{l@{\hspace{1.5cm}}D{.}{,}{3.2}D{.}{,}{1.2}D{.}{,}{2.3}}
		\toprule
		\textbf{Nastavení mapy a EA}\\
		\midrule
		Roboti: & Refaktor, Worker, Scout\\
		Počty robotů: & R-2,\ W-3,\	 S-4 \\
		Počet generací: & 2000\\
		Počet iterací map & 1500\\
		Velikost generace(DE) & 100\\
		\bottomrule
		\multicolumn{2}{l}{}
	\end{tabular}
	\par 
	\begin{tabular}{l@{\hspace{1.5cm}}D{.}{,}{3.2}D{.}{,}{1.2}D{.}{,}{2.3}}
		\toprule
		\textbf{Fitness funkce}\\
		\midrule
		Hodnota přeměného paliva & 1000\\ 
		Hodnota paliva v nádržích & 1000\\
		Ostatní hodnoty: & 0\\
		\toprule
		\textbf{Objekty na mapě}\\
		\midrule
		Počet minerálů: & 400\\
		Počet překážek & 100\\
		Počet paliva & 0\\
		\bottomrule
		\multicolumn{2}{l}{}
	\end{tabular}
	\caption{Mineral Refaktor Worker kooperace - nastavení experimentu}
	\label{tab04:MineralFullCoop}
\end{table}
Křivka opět ukazuje  vzrůst fitness, hodnota se ustaluje kolem $4.1 \cdot10^7$ . Při přeměně minerálu vznikne 100 jednotek paliva. Roboti dostaly na začátku simulace více paliva než potřebovali na běh, tohoto paliva jim zbylo 8 500 jednotek (ve fitness $0.85\cdot 10^7$ ). Takže výsledná fitness tohoto experimentu odpovídá přibližně 300 zpracovaným minerálů. Nejlepšímu chování se budeme věnovat v rámci výsledků experimentů \ref{subsec:MineralResult}.
\clearpage
\begin{figure}[h]\centering
	\includegraphics[width=\columnwidth]{../img/MineralMap/MineralFullCoop}
	\caption{Mineral Refaktor Worker kooperace -  průběh fitness DE}
	\label{obr04:MineralFullCoop}
\end{figure}
Modrá křivka odpovídá průměrné hodnotě fitness, světle modře je pak vybarvena oblast daná průměrem $\pm$ směrodatná odchylka fitness. Generace (osa y) jsou sdruženy po 10. Fitness je vydělena $10^7$, aby popisky nezabíraly příliš mnoho místa. 
\clearpage
\subsection{Výsledky Experimentu}
\label{subsec:MineralResult}
Ač předchozí graf fitness \ref{obr04:MineralFullCoop} vypadal velmi optimisticky, výsledky \ref{tab04:MineralStat} z vícero běhů na různých mapách jim neodpovídají . 
\par 
Po dlouhém pozorování optimalizovaných jedinců jsem dospěl k závěru, že optimalizace probíhá ovšem u takto složitých úkolů se soustředila pouze na Refaktor robota.  A to z jednoduchého důvodu, protože sám o sobě dokázal nejvíce ovlivňovat fitness funkci pro optimalizaci byl tutíž nejvhodnější volbou. Dále jelikož úkony spojené s hlavním cílem scénářem byly velmi složité, tak se chování optimalizovalo v závislosti na aktuálně vygenerované mapě. Každému experimentu odpovídá jedna náhodně generovaná mapa kvůli jednoznačnosti ohodnocení fitness. Nicméně v jednodušších podúkolech dokázala DE vyevolvovat univerzální a úspěšné jedince.
\par 
V tomto případě ani nepřikládám obrázek s koncem běhu tohoto chování, neboť je velmi podobný tomu z  náhodného běhu. Všechny chování nejlepší generace si lze prohlédnout v programu na přiloženém CD. 
 \unsure{Nemám uloženou mapu na které byl optimalizován, což je asi problém.}
\begin{table}[h]\centering   
	\begin{tabular}{l@{\hspace{1.5cm}}D{.}{,}{3.2}D{.}{,}{1.2}D{.}{,}{2.3}}
		\toprule
		& \mc{} & \mc{}\\
		\textbf{Inicializační nastavení:}  \\
		\midrule
		Výška: & 800\\ 
		Šířka: & 1200\\
		Počet iterací: & 1500\\
		Počet minerálů: & 400\\
		Počet překážek: & 50 \\
		Počet Scout robotů: & 4\\
		Počet Worker robotů: & 3\\
		Počet Refaktor robotů: & 2\\
		Inicializační palivo: & 1500\\
		Spotřeba paliva robotů: & 1/kolo\\
		\bottomrule
		\multicolumn{2}{l}{}
	\end{tabular}
	\caption{Mineral Scene - nastavení mapy pro testovací experiment}
\end{table}
\begin{table}[h]\centering   
	\begin{tabular}{l@{\hspace{1.5cm}}D{.}{,}{3.2}D{.}{,}{1.2}D{.}{,}{2.3}}
		\toprule
		& \mc{} & \mc{}\\
		\textbf{Výsledky} \\
		\bottomrule
		Překážky v mapě & 49.98\\
		Objevené překážky & 2.51\\
		Minerálů v mapě & 393.7\\
		Z toho nalezených & 17.25\\
		Palivo v kontejnerech & 0.02\\ 
		Zbylé palivo v nádržích & 59\\ 
		Kolize & 9\\
		\multicolumn{2}{l}{}
	\end{tabular}
	\caption{Mineral Scene - výsledky simulace nejlepšího jedince, průměr ze 100 simulací testovacího experimentu}
	\label{tab04:MineralStat}
\end{table}
\clearpage
\section{Competitive Scene}
Poslední ze scénářů se týká soutěžení dvou týmů (hejn), kteří se snaží to druhé zničit. Úspěšnost týmu je dána zachovanými jednotkami zdraví robotů a uděleným poškozením do nepřátelské skupiny robotů. 
 \par
 V mém Competitive Scene scénáři proti sobě stojí dvě hejna, která čítají dohromady 9 robotů. Mimo roboty v mapě jsou také náhodně rozmístěny překážky, který se musí roboti vyhýbat. Týmy začínají v první a poslední čtvrtině mapy, kde startují na náhodné pozici. 
 \par 
Hejno se skládá ze dvou druhů robotů. Roli průzkumníka zastává \textit{Scout robot}, který je malý a rychlý. I Scout robot může způsobovat poškození, ovšem pouze jednu pětinu oproti druhému robotovi, pro kterého používám v rámci tohoto scénáře název \textit{Fighter robot}. Fighter robot se pohybuje méně obratně, jelikož je dvakrát větší a pomalejší. 
\par
V tomto scénáři maji všichni roboti k dispozici rádiové signály s kódy nula až 4, tudíž je čistě na optimalizaci, jak je využije. Competitive Scene obsahuje základní rojové scénáře jako je komunikace, vyhybání překážkám, hledání v mapě, apod. Tentokrát budou mít roboti obtížnější hledání cílů, protože budou pohyblivé a také jim mohou způsobit poškození. 
\par 
Popis jednotlivých entit v mapě se nachází u obrázku \ref{obr04:CompetitiveSceneRandomStart}
\clearpage
\begin{figure}[p]\centering
	\includegraphics[width=\columnwidth]{../img/todo}
	\caption{Příklad Competitive Scene mapy: start náhodného chování}
	\label{obr04:CompetitiveSceneRandomStart}
\end{figure}
\clearpage 

\subsection{Roboti}
Jak už jsem zmínil, ve Competitive Scene scénáři se objevují dva druhy robotů. V každém týmu se objevují 4 Scout roboty a 5 robotů typu Fighter. Pro účely správného vyhodnocování fitness zůstává po celou dobu běhu experiment nepřátelský tým řízený stejným chováním. Typicky je na začátku pro nepřátelským vygenerováno náhodné chování. Podoba jedinců opět odpovídá předchozím experimentům i v kontextu nepřátelských robotů. V implementaci má aktuálně optimalizovaný tým označení tým 1 a protivník tým 2. 
Teď se podíváme na jednotlivé roboty.
\subsubsection{Fighter Scout robot}
Roli lehkého útočníka ve scénáři zastává Scout robot. Jedná se o malého a obratného robota. Dokáže způsobit poškození za 100 bodů zdraví. K poškození slouží úsečkový efektor (v rámci implementace pojmenovaný \textit{Weapon}).  Efektory Weapon fungují na stejném principu jako ostatní úsečkové efektory, pro udělení poškození musí kolidovat s jiným robotem. V rámci tohoto experimentu je možné udílet poškození i robotům z vlastního týmu.  Jeho podrobnou specifikaci poskytuje tabulka \ref{tab04:CompetiveScout}
\subsubsection{Fighter robot}
Fighter robot je navržen jako těžký bitevník. Oproti Scout robotovi je větší a pomalejší. Nejmarkantnější rozdíl tvoří body zdraví a síla útoku. Fighter robot způsobuje pětkrát vyšší poškození a disponuje trojnásobkem bodů zdraví.  Navíc může útočit na jednou až čtyřmi zbraněmi najednou, zatímco Scout robot pouze třemi. I on může útočit do vlastních řad. 
Jeho konkrétní specifikaci lze nalézt v tabulce \ref{tab04:CompetitiveFighter}.
\begin{table}[h]\centering
	\begin{tabular}{l@{\hspace{1.0cm}}D{.}{,}{2.2}D{.}{,}{2.2}D{.}{,}{2.3}}
		\toprule
		\textbf{Fighter Scout Robot} \\
		\midrule
		Tvar: & Kruh\\
		Poloměr: & 2,5\\
		Body zdraví: &500\\
		Namespace: & CompetitiveRobots\\
		Název: & FighterScoutRobotMem \\
		\midrule
		\textbf{Efektory} \\
		\midrule
		Motor: & Dvou kolečkový \\
		Maximální rychlost: & 3 \\
		Kód rádiového signálu: & 0,1,2\\
		Poloměr signálu: & 200\\
		Počet paměťových slotů: &10 \\
		Obsah slotu: & float\\
		Počet zbraní: & 3\\
		Dosah zbraní: & 10\\
		Orientace zbraní: &  0^\circ, \pm 45^\circ\\
		Útok zbraní: & 100\\
		\midrule
		\textbf{Senzory} \\
		\midrule
		Počet line senzorů: &  3\\
		Délka line senzorů: & 70\\
		Orientace l. senzorů: & 0^\circ, \pm45^\circ\\
		Poloměr rádiového přijímače: & 100 \\
		Poloměr type senzoru: & 50\\
		Počet touch senzorů: & 3 \\  
		Lokátorový senzor\\ 
		\bottomrule
		\multicolumn{2}{l}{}
	\end{tabular}
	\caption{Competitive Scene - Fighter Scout robot specifikace }
	\label{tab04:CompetiveScout}
\end{table}
\clearpage

\par  
\begin{table}[h]\centering
	\begin{tabular}{l@{\hspace{1.0cm}}D{.}{,}{2.2}D{.}{,}{2.2}D{.}{,}{2.3}}
		\toprule
		\textbf{Fighter Scout Robot} \\
		\midrule
		Tvar: & Kruh\\
		Poloměr: & 5\\
		Namespace: & CompetitiveRobots\\
		Název: & FighterRobotMem \\
		Body zdraví: & 1500\\
		\midrule
		\textbf{Efektory} \\
		\midrule
		Motor: & Dvou kolečkový \\
		Maximální rychlost: & 1,5 \\
		Kód rádiového signálu: & 0,1,2\\
		Poloměr signálu: & 200\\
		Počet paměťových slotů: &10 \\
		Obsah slotu: & float\\
		Počet zbraní: & 4\\
		Orientace zbraní: &  0^\circ, \pm 45^\circ, 180^\circ\\
		Útok zbraní: & 500\\
		\midrule
		\textbf{Senzory} \\
		\midrule
		Počet line senzorů: &  3\\
		Délka line senzorů: & 70\\
		Orientace l. senzorů: & 0^\circ,\pm45^\circ\\
		Poloměr rádiového přijímače: & 100 \\
		Poloměr type senzoru: & 50\\
		Počet touch senzorů: & 3 \\  
		Lokátorový senzor\\ 
		\bottomrule
		\multicolumn{2}{l}{}
	\end{tabular}
	\caption{Competitive Scene - Fighter robot specifikace }
	\label{tab04:CompetitiveFighter}
\end{table}
\clearpage
\subsection{Vyhodnocování Fitness}
Hlavní cíl Competitive Scene scénáře se skládá z mnoha úkonů. Proto jsem ho stejně jako v předchozích scénářích rozdělil na menší metaúkoly. Tradičně jsem začal s učením pohybu a vyhybáním se překážkám. Poté jsem pokračoval přes uchování, co nejvíce životních bodů, až po způsobení maximálního poškození protivníka. Fitness funkce má podobu váženého součtu stejně jako u Wood Scene a Mineral Scene. A používal jsem k optimalizaci pouze DE. 
\par 
Na konci simulace mapy jsou roboti oceněni za:  
\begin{enumerate}
	\item \textit{nalezené překážky} - překážky o které zavadil line sensor
	\item \textit{zabité roboty} - mrtvé roboty nepřátelského týmu
	\item \textit{udělený útok} - útok udělený nepřátelským robotům 
	\item \textit{zbývající životy} - životy zbývající aktuálním robotům
\end{enumerate}
Trestáni za:
\begin{enumerate}
	\item \textit{kolize} - počet pokusů o pohyb při kterém by došlo ke kolizi 
\end{enumerate}

\subsection{Podúkoly}
V prvních podúkolech opět tradičně figurují roboti odděleně. Jejich princip je shodný, protože roboti mají velmi podobné funkce. 
\clearpage

\subsubsection{Scout chůze - nastavení experimentu}

\begin{table}[h]\centering   
	\begin{tabular}{l@{\hspace{1.5cm}}D{.}{,}{3.2}D{.}{,}{1.2}D{.}{,}{2.3}}
		\toprule
		\textbf{Nastavení mapy a EA}\\
		\midrule
		Roboti: & Scout-5 \\
		Počet generací: & 1000\\
		Počet iterací map & 1000\\
		Velikost generace(DE) & 100\\
		\bottomrule
		\multicolumn{2}{l}{}
	\end{tabular}
	\par 
	\begin{tabular}{l@{\hspace{1.5cm}}D{.}{,}{3.2}D{.}{,}{1.2}D{.}{,}{2.3}}
		\toprule
		\textbf{Fitness funkce}\\
		\midrule
		Hodnota nalezeného minerálu &  100 \\
		Ostatní hodnoty: & 0\\
		Počet minerálů: & 400\\
		Počet překážek & 100\\
		Počet paliva & 100\\
		\bottomrule
		\multicolumn{2}{l}{}
	\end{tabular}
	\caption{Scout chůze - nastavení experimentu}
	\label{tab04:CompetitiveWalk}
\end{table}
\redo{TODO}
\clearpage
\subsection{Výsledky Experimentu}
\redo{TODO}
\begin{table}[h]\centering   
	\begin{tabular}{l@{\hspace{1.5cm}}D{.}{,}{3.2}D{.}{,}{1.2}D{.}{,}{2.3}}
		\toprule
		& \mc{} & \mc{}\\
		\textbf{Inicializační nastavení:}  \\
		\midrule
		Výška & 800\\ 
		Šířka & 1200\\
		Počet iterací & 10000\\
		Počet stromů & 400\\
		Počet Scout robotů & 5\\
		Počet Worker robotů & 4\\
		\bottomrule
		\multicolumn{2}{l}{}
	\end{tabular}
	\caption{WoodScene - nastavení mapy pro testovací experiment}
\end{table}
\begin{table}[h]\centering   
	\begin{tabular}{l@{\hspace{1.5cm}}D{.}{,}{3.2}D{.}{,}{1.2}D{.}{,}{2.3}}
		\toprule
		& \mc{} & \mc{}\\
		\textbf{Výsledky} \\
		\bottomrule
		Zpracované dřevo zanechané v mapě & 225.22\\
		Stromy v mapě & 156.22\\
		Z toho nalezené & 52.84\\
		Dřevo v kontejnerech & 18.48\\
		Uskladněné dřevo & 17.56\\
		\multicolumn{2}{l}{}
	\end{tabular}
	\caption{WoodScene - výsledky simulace nejlepšího jedince, průměr ze 100 simulací testovacího experimentu}
	\label{tab04:CompetitiveStat}
\end{table}
\newpage
\begin{figure}[p]\centering
	\includegraphics[width=\columnwidth]{../img/todo}
	\caption{Nejlepší jedinec - 10000 iterací simulace}
	\label{obr04:CompetitiveBestEnd}
\end{figure}
\begin{figure}[p]\centering
	\includegraphics[width=\columnwidth]{../img/todo}
	\caption{Náhodný jedinec - 10000 iterací simulace}
	\label{obr04:CompetitiveRandom end}
\end{figure}
\clearpage
\section{Shrnutí}

\chapter*{Závěr}
\addcontentsline{toc}{chapter}{Závěr}
V rámci této práce byl implementován kompletní 2D simulátor umožňující simulovat chování robotických hejn, simulátor umožňuje hejnu, abys se skládalo z rozličných jedinců. Tento simulátor také obsahuje metody EA, konkrétně DE a ES jako prostředky pro optimalizaci řízení robotických hejn. Pro vizualizaci těchto chování vznikl program umožňující prohlížení už optimalizovaných chování. Simulace byla také optimalizována a bylo použito paralelního zpracování. 
\par
Za pomocí těchto programů byla otestováno evolucí generované ovládání heterogenních hejn v rámci tří rozličných scénářů. Úkoly v těchto scénářích byly tvořeny z tradičních problémů robotických hejn jako je shlukování, pohyb v prostředí, rozptylování, apod. Dohromady tvořily scénáře komplexní problémy s netriviálním řešením. 
\par
Během prvního experimentu byly porovnány dva evoluční algoritmy pro optimalizaci neuronových sítí pro řízení heterogenních  swarmů, konkrétně se jednalo o DE a ES. V rámci toho experimentu bylo provedeno několik testovacích běhů pro zmíněné srovnání. DE se ukázala jako lepší varianta pro optimalizaci heterogenního robotického hejna. 
\par 
V dalších dvou experimentech se bylo potvrzeno, že tuto metoda lze použít jako vhodný nástroj pro optimalizaci jednoduchých chování hejna. Ovšem objevili se i limity spojené s obtížnými úkony, kde DE nepracovala uspokojivě. Bylo navrženo řešení těchto nedostatků, ovšem z časových důvodů nebylo vyzkoušeno. 

\subsection*{Možná rozšíření}
Tato práce poskytuje řadu zajímavých možností k rozšíření ať už se jedná o řešení nevyzkoušených metod na neúspěšných scénářích či porovnání s jinými metodami.\par
V rámci nedostatků scénářů Mineral Scene a Competitive Scene by bylo žádoucí vyzkoušet evoluční algoritmy, které umí generovat složitější neuronové sítě. Jedná se především o algoritmy odvozené z NEAT rodiny. Otestovat je na těchto obtížných scénářích.
\par
Pro porovnání úspěšnosti evolučního algoritmu s jinýmu učícími algoritmy by bylo vhodné otestovat DE a ES s tradičním \textit{backpropagation} algoritmem pro neuronové sítě. 
\par
V neposlední řadě by zajímavou cestu tvořilo přenesení vyvinutých chování na fyzické robot. Což by ale vyžadovalo velké úpravy na simulátoru, protože simulátor značně zjednodušuje akce v prostředí. Navíc by bylo třeba přidat šumy prostředí neboť senzory jsou nimi v reálném světě značně zkresleny. 

%%% Seznam použité literatury
\include{literatura}

%%% Obrázky v bakalářské práci
%%% (pokud jich je malé množství, obvykle není třeba seznam uvádět)
\listoffigures

%%% Tabulky v bakalářské práci (opět nemusí být nutné uvádět)
%%% U matematických prací může být lepší přemístit seznam tabulek na začátek práce.
\listoftables

%%% Použité zkratky v bakalářské práci (opět nemusí být nutné uvádět)
%%% U matematických przkratek}ací může být lepší přemístit seznam zkratek na začátek práce.
%%% \chapwithtoc{Seznam použitých 

%%% Přílohy k bakalářské práci, existují-li. Každá příloha musí být alespoň jednou
%%% odkazována z vlastního textu práce. Přílohy se číslují.
%%%
%%% Do tištěné verze se spíše hodí přílohy, které lze číst a prohlížet (dodatečné
%%% tabulky a grafy, různé textové doplňky, ukázky výstupů z počítačových programů,
%%% apod.). Do elektronické verze se hodí přílohy, které budou spíše používány
%%% v elektronické podobě než čteny (zdrojové kódy programů, datové soubory,
%%% interaktivní grafy apod.). Elektronické přílohy se nahrávají do SISu a lze
%%% je také do práce vložit na CD/DVD. Povolené formáty souborů specifikuje
%%% opatření rektora č. 23/2016.
\chapwithtoc{Obsah přiloženého CD}
\begin{itemize}
	\item \textbf{bin} - obsahuje spustitelné soubory (hlavní konzolovou aplikaci, program pro vizualizaci) a všechny potřebné programy pro spuštění
	\item \textbf{doc} - obsahuje uživatelskou i programátorskou dokumentaci
	\item \textbf{data} - obsahuje příklady konfigurací experimentů a výsledky jednotlivých experimentů uvedených v této práci 
	\item \textbf{src} - obsahuje zdrojové kódy obou zmíněných projektů a jejich testů  
\end{itemize}

\openright
\end{document}
