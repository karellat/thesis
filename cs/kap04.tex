%%% Kapitoly
\chapter{Experimenty}
\section{Úvod}
Všechny práce zmíněné v úvodní kapitole, sice používají evolučních algoritmů k vytvoření řízení chování robotického swarmu s pouze jedním druhem robotů. Cílem této práce je zmapovat použití jednoduchých evolučních algoritmů i pro heterogenní swarmy. Následující části mají přiblížit obecný postup při hledání optimálního chování hejn, stručně popsat jednotlivé experimenty a poté se věnovat podrobněji každému experimentu zvlášť. Nejvíce se rozepíši o prvním experimentu, protože ten sloužil jako testující pro rozličné postupy a podle něj jsem přistupoval i k ostatním experimentům.  

\section{Použité algoritmy}
Vybral jsem dva evoluční algoritmy, které se zdáli při prvotních testech nejvíce perspektivní. Při evoluci homogenních robotů se nejčastěji používají Evoluční Strategie jako jeden z optimalizačních algoritmů, také proto jsem je zvolil jako jeden z využívaných algoritmů. Druhá volba padla na o něco méně agresivní optimalizaci v podobě Diferenciální Evoluce. Oba algoritmy jsou popsány v kapitole o evolučních algoritmech. \par 
Při prvotních zkušebních optimalizací se ukázalo, že optimalizovat rovnou celý cíl jednotlivých scénářů není příliš slibné. Po konzultaci s vedoucím, jsem se rozhodl rozdělit vždy cíl na jednotlivé podúkoly hlavního cíle scénaře. Jednotlivé podúkoly neřeší vždy všichni roboti najednou, ale některé jen homogenní skupina. Nicméně nejpozději v posledním podúkolu už jsou evolvovány společně.
\par
Zkoušel jsem také testovat, roboty s paměťovými sloty oproti těm bez. Úspěnost a konvergence robotů s pamětí se znatelně se učil pomaleji. Roboty bez slotů se nebyly schopni přiblížit k výsledkům těm s pamětí ani při velkém počtu generací. Z tohoto důvodu všichni roboti mají připojeno alespoň 10 paměťových slotů. Paměťový slot zároveň chová jako sensor i jako efektor. 
\par 
Jako samozřejmostí u každého robota najdeme line sensory, touch sensory, dvou-kolečkový motor, lokátorový sensor. \par 

\subsection{Diferenciální Evoluce}
TODO: Nastavení parametrů, varianta, praktická implementace

\subsection{Evoluční strategie}
TODO: Nastavení parametrů, varianta, praktická implementace

\section{Experimenty}
Pro testování jsem zvolil tři rozličné scénáře. Hlavní motivací bylo jednotlivé úkoly pro hejno udělat udělat komplexnější, aby každá skupina robotů uměla řešit pouze část ze zadání úkoli. Také jsem se snažil, aby se scénáře blížili reálným situacím v dnešním každodenním světě.\par 
\textbf{Pracovní názvy}
\begin{enumerate}
    \item Wood Scene
    \item Mineral Scene
    \item Competitive Scece
\end{enumerate}

\subsection{Wood Scene}
Tento scénář je analogií pro kácení lesa, kdy se roboti snaží maximalizovat množství zpracované dřevo na předem vyznačené ploše. První robot plní úkol objevovávání a kácení stromu, ale neumí je převážet, v mém frameworku se nazývá Wood Scout. Oproti tomu druhý robot má vlastní kontajner na objekty, také je umí zvedat a následně pokládat. Ve frameworku pojmenovaný Wood Worker. Ovšem neumí stromy zpracovávat. Jedná se tedy o úkol typu najdi označ a převez.

\subsection{Mineral Scene}
Jedná se o scénář reprezentující sběr surovin pro výrobu paliva a jeho následné využití. Figurují zde 3 rozliční roboti, všichni potřebují pro pohyb  dané množství paliva. Úspěšnost daného hejna se měří množstvím paliva. Nejmenší robot(Mineral scout) disponuje pouze sensory k exploraci prostředí a rádiovým vysílačem pro komunikaci se skupinou. Robot prostřední velikosti(Mineral Worker) se pohybuje o něco pomaleji než Mineral Scout, ale umí přesouvat objekty i více najednou. Robot pro přeměnu minerálu(,suroviny na výrobu paliva,) označen ve frameworku jako Mineral Refactor se přemisťuje nejpomaleji, má možnost přeměnit minerál na palivo. Tento scénář si bere jako inspiraci strategické hry a hypotetické přežití robotů na cizí planetě, kde si budou muset obstarat vlastní 
nerostné suroviny pro běh.

\subsection{Competitive Scene}
Poslední ze scénářů se týká soutěže dvou týmů(hejn) ve kterých figurují jeden malý průzkumný robot(Competitive Scout) a jeden vetší bojový robot(Competitive Fighter). Úspěšnost týmu je dána zachovanými jednotkami zdraví robotů a uděleným poškozením do nepřátelské skupiny robotů. Competitive Scout se pohybuje značně rychleji než Competitive Fighter, ale uděluje menší poškození. Což lze opět vztáhnout na chování rozdílných skupin nepřátel např. ve strategických hrách, kde se jejich chování adaptuje, co nejlépe na dané prostředí. 

\section{WoodScene experiment}
Cílem tohoto scénáře je shromáždit uprostřed mapy, co nejvíce zpracovaného dřeva. Plocha pro skládání dřeva je označena rádiovým signálem s hodnotou signálu 2. V experimentu se dohromady celkem vyskytuje 9 robotů dvou různých druhů. Roboti jsou na začátku simulace umístěni uprostřed mapy na skládácím prostoru, po-té následuje 2000 iterací simulace mapy. 

\subsection{Roboti}

\subsubsection{Scout robot}
Jedná se o robota, který má na starosti průzkum mapy a kácení nalezených stromů. Pro komunikaci s ostatními roboty má možnost vysílat rádiový signál s hodnotou 0. Oproti Worker robotovi je menší, rychlejší, jeho sensory mají větší dosah, navíc proti němu disponuje type sensorem a refaktorem nalezených stromů. Type sensor představuje formu radaru, říká robotovi s jakou četností se vyskytují v dosahu sensoru. Refaktor reprezentuje techniku kácení mění strom na dřevo. 
\par 
TODO: Tabulka velikostí

\subsubsection{Worker robot}
Worker robot se stará o transport objektů na mapě. Ke komunikaci využívá signálů s kódem 1. Sebrané objekty ukládá do kontejneru, kam se vejde 5 entit. Zvedání a pokádání probíhá skrze efektor Picker. 
\par 
TODO: Tabulka velikostí

\subsection{Vyhodnocování Fitness}
Fitness funkce pro ohodnocení WoodScene scénáře probíhá vždy až nakonci simulace. Ikdyž se úspěšnost v podúkolech  vždy posuzuje jinak, celou fitness funkci lze shrnout do následujícího cílů. Roboti jsou odměňováni za: 
\begin{enumerate}
        \item kolize = počet pokusů o pohyb při kterém by došlo ke kolizi 
        \item nalezené stromy = stromy o které zavadil line sensor 
        \item pokácené stromy = stromy, které refaktor změnil 
        \item sebrané dřevo = zpracované dřevo, které mají roboti unitř kontajnerů 
        \item uskladněné dřevo = dřevo, které dovezli na vyznačené místo 
\end{enumerate}

\subsection{Podúkoly} 
Pro oba použité algoritmy jsem používal stejné úlohy pro naučení robotů postupně težších a těžších cílů. Hlavně také, abych mohl porovnat oba evoluční algoritmy. TODO: Bojím se, že pro ES to nebude stačit 
\begin{enumerate}
        \item vygenerování robotů = Na záčátku je vygenerováno chování robotů naprosto náhodně. Pro každého robota, je vygenerována náhodná jednovrstvá neuronová síť. 
        \item učení chůze = Pro oba roboty je velmi důležité, aby se pohybovali bez kolizí po celé mapě a objevovali, co největší prostor. Roboti jsou vyvíjeni odděleně a fitness se soustředí na počet kolizí(záporným ohodnocením) a na nalezené stromy(kladným ohodnocením).
        \item težba stromů = Scout roboty, kteří se už obstojně po mapě pohybují, je třeba naučit kácet stromy. Proto dalším podúkol cílí ve fitness funkci na počet pokácených stromů. Nicméně stále také na počet stromů nalezných. 
        \item převoz dřeva = Správně pohybující chceme naučit sbírat vytěžené dřevo. Fitness hodnotí počet sebraného dřeva, případně i uskladněné dřevo. Na těchto mapách jsou už na začátku připraveny pouze entity zpracovaného dřeva.
        \item kooperace = V posledním experimentu, se hodnotí pouze sebrané a uskladněné dřevo. A evolvují se oba druhy robotů současně. 
\end{enumerate}

\subsection{Nastavení EA}
\subsubsection{Diferenciální Evoluce}
Pro differenciální evoluci jsem zvolil 100 jedinců jako velikost populace, kteří procházejí přes 1000 generací. Mezi nejtěžší připravy evolučních algoritmů patří volba parametrů evoluce u DE se konkrétně jedná o pravděpodnost křížení(CR) a váhu diference(F) (, v angličtině se nazývá Differential weight). Po první testech jsem uvážil $F = 0.8 $ a $CR = 0.5$ jako nejlepší volbu. 
\subsubsection{Evoluční Strategie}
U Evolučních Strategiích, které mají trochu jinou formu mutací, budu evolvovat 8 jedinců. Tyto jedinci projdou 200 generací a velikost mutačního kroku  v každé generaci je právě 20. Varianta $(\sigma,\lambda)$ se ukázala jako vhodnější volba, neaplikuji tedy elitimus, při testech se stávalo, že se jedinec na začátku zasekl v lokálním optimu a dál se vůbec nevyvíjel. Jako learnig rate neboli alpha se osvědčila hodnota 0.05 a rozptyl u šumů normálního rozdělení sigma roven 0.1. TODO: Asi bude třeba měnit u jednotlivých experimentů.
\subsection{Výsledek Experimentu}
