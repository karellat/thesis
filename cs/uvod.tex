\chapter*{Úvod}
\addcontentsline{toc}{chapter}{Úvod}

\title{Úvod}
Využití robotických hejn (robotic swarms) patří mezi rentabilní metody pro řešení složitějších úkolů. Existuje řada studií potvrzujících, že velký počet jednoduchých robotů dokáže plně nahradit komplexnější jedince. Dostatečná velikost hejna umožní řešení úloh, které by jednotlivec z hejna provést nesvedl. Navíc robotické hejno přináší několik výhod díky kvantitě jsou odolnější proti poškození i po zničení některých z nich zbytek robotů pokračuje v plnění cílů. Dále výroba jednodušších robotů vychází levněji než komplexních jedinců, což přináší nezanedbatelnou výhodu pro práci v nebezpečném prostředí. Hejno také může pokrývat vícero různých úkolů než specializovaný robot, který bude při plnění úkolů, lišících se od typu úloh zamýšlených při konstrukci, mnohem více nemotorný a nejspíše pomalejší. Hejno pokryje větší plochu při plnění úkolů. 
\par
Existuje mnoho aplikací robotických hejn, vetšinou se používají v úlohách týkajících se průzkumu a mapování prostředí, hledání nejkratších cest, nasazení v nebezpečných místech. Jako příklad můžeme uvést asistenci záchranným složkám při požáru \citep{fireRobots}. Mnoho projektů zabývající se řízením robotického hejna se inspiruje přírodou, používá se analogie s chováním mravenců a jiného hmyzu \citep{PheroRobot}. Objevují se i hardwarové implementace chování hejn, zmiňme projekty Swarm-bots \citep{swarmBots}, Colias \citep{Colias}  
\par 
Elementárnost senzorů i efektorů jednotlivých robotů vybízí k použití evolučních algoritmů, jelikož prostor řešení je rozlehlý a plnění cílů lze vhodně ohodnotit. Vzniklo několik vědeckých prací popisující problematiku tohoto tématu \citep{ENovel} \citep{geneticSwarm}.
\section*{Cíl práce}
Všechny zmíněné práce používají pro tvorbu řídicích programů evoluční algoritmy (EA) a pracují pouze s homogenními hejny. Cílem této práce je vyzkoušet využití EA na generování chování hejna heterogenních robotů, tedy skupiny agentů, ve které se objevuje několik druhů jedinců a společně plní daný úkol. V rámci práce byl vytvořen program pro simulaci různých scénářů. Pro otestování jejich úspěšnosti v rámci EA, byly zvoleny 3 odlišné scénáře, ve kterých se objevují 2-3 druhy robotů.
\section*{Struktura práce}
Rozdělení práce je následující. První kapitola je věnována obecnému úvodu do tématiky evolučních algoritmů, kde se podrobněji věnuji evolučním strategiím a diferenciální evoluci, protože oba tyto postupy implementuji v programu pro řešení scénářů. Druhá kapitola se zabývá představením robotického hejna, základním principům a několika konkrétnějším aplikacím. Ve třetí kapitole je nastíněno fungování simulátoru, přiložené dokumentace obsahuje podrobnější informace o implementaci simulátoru. Všechny provedené experimenty se všemi detaily obsahuje čtvrtá kapitola.