\chapter*{Závěr}
\addcontentsline{toc}{chapter}{Závěr}
V rámci této práce byl implementován kompletní 2D simulátor umožňující simulovat chování robotických hejn, simulátor umožňuje hejnu, abys se skládalo z rozličných jedinců. Tento simulátor také obsahuje metody EA, konkrétně DE a ES jako prostředky pro optimalizaci řízení robotických hejn. Pro vizualizaci těchto chování vznikl program umožňující prohlížení už optimalizovaných chování. Simulace byla také optimalizována a bylo použito paralelního zpracování. 
\par
Za pomocí těchto programů byla otestováno evolucí generované ovládání heterogenních hejn v rámci tří rozličných scénářů. Úkoly v těchto scénářích byly tvořeny z tradičních problémů robotických hejn jako je shlukování, pohyb v prostředí, rozptylování, apod. Dohromady tvořily scénáře komplexní problémy s netriviálním řešením. 
\par
Během prvního experimentu byly porovnány dva evoluční algoritmy pro optimalizaci neuronových sítí pro řízení heterogenních  swarmů. Zdá se, že DE jsou lepším optimalizačním nástrojem oproti ES. Pro potvrzení této domněnky by ale bylo potřeba provést více experimentů.
\par 
V dalších dvou experimentech bylo potvrzeno, že tuto metodu lze použít jako vhodný nástroj pro optimalizaci jednoduchých chování hejna. Ovšem objevily se i limity spojené s obtížnými úkony, kde DE nepracovala uspokojivě. Bylo navrženo řešení těchto nedostatků, ovšem z časových důvodů nebylo vyzkoušeno. 

\subsection*{Možná rozšíření}
Tato práce poskytuje řadu zajímavých možností k rozšíření ať už se jedná o řešení nevyzkoušených metod na neúspěšných scénářích či porovnání s jinými metodami.
\par
Jako vhodné rozšíření se nabízí potvrzení porovnání DE a ES dostatečným množstvím opakování, k čemuž bude potřeba velká výpočetní síla a počet strojů.
\par
V rámci nedostatků scénářů Mineral Scene a Competitive Scene by bylo žádoucí vyzkoušet evoluční algoritmy, které umí generovat složitější neuronové sítě. Jedná se především o algoritmy odvozené z NEAT rodiny. Otestovat je na těchto obtížných scénářích.

\par
V neposlední řadě by zajímavou cestu tvořilo přenesení vyvinutých chování na fyzické roboty. Což by ale vyžadovalo velké úpravy na simulátoru, protože simulátor značně zjednodušuje akce v prostředí. Navíc by bylo třeba přidat šumy prostředí neboť senzory jsou jimi v reálném světě značně zkresleny. 