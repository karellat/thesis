\chapter{Evoluční algoritmy}
Genetické programování je kolekce metod v oboru umělé inteligence. V  GP vyvijíme počítočové populaci počítačových programů. Kde počítačový program představují geny(vektor čísel), hledaným výsledkem je nakonec program reprezentovaný konkrétním genem, který vhodně plní předem definovaný úkol. Prohledávání prostoru všech možných genů zajišťují evoluční algoritmy(též zvané genetické algoritmy), které populace genů modifikují, ohodnocují, dávají tím vzniknout nové lepší populaci. Tento proces se opakuje, dokud není splněna koncová podmínka. Metoda tedy zahrnuje vytvoření reprezentace programu jako vektoru čísel a dále hledání úspěšného genu v prostoru všech možných genů \citep{field}. Tyto metody se inspirují teorií evoluce(přirozený výběr), stejně jako v přírodě jedná se o náhodný proces, proto tedy není garantovaný výledek a GP řadíme k stochastickým metodám. 
\section{Evoluční algoritmy}
Začněme úvodem z historie Evolučních algoritmů podle knihy \citep{MitchellBook}. EA začaly být zkoumány v 50. a 60. letech 19. století nezávisle řadou vědců. Společným základem všech teorií byla evoluce populace kandidátů na řešení daného problému a jejich následná úprava způsoby hromadně nazývány jako genetické operátory, například mutace genů, přirozená selekce úspěšnějších řešení. \par 
V 60. letech se už objevují konkrétnější metody. Rechenberg (1965, 1973) představuje Evoluční strategie, metoda optimalizující parametry v reálných číslech, jejich použití pro letadlová křídla. Fogel, Owens, Walsh zveřejňují \textit{evolutionary programming}(evoluční programování), technika využívající k reprezentaci kandidátů konečný automat(s konečným počtem stavů), který byl vyvíjen mutací přechodů mezi stavy a následnou selekcí. \textit{Genetické algoritmy} vynalezl Holand v 60. letech a následně se svými studenty a kolegy z Michiganské Univerzity implementoval, oproti ES a EP nebylo hlavním cílem formovat algoritmus pro řešení konkrétních problémů, ale přenos obecného mechanismu evoluce jako metody aplikovatelné v informatickém světě. Princip GA spočívá v transformaci populace chromozonů(př. vektor 1 a 0) v novou populaci pomocí genetických operátorů křížení, mutací a inverze. V 1975 v knize \textit{Adaptation in Natural and  Artificial Systems} \citep{HolandBook} definoval genetický algoritmus jako abstrakci biologické evoluce spolu s teoretickým základem jejich používání. Ovšem někteří vědci používají pojem GA i ve významech hodně vzdálených původní Holandově definici. Tyto tři větve GA, ES, EP tvoří základ Genetického programování. 
