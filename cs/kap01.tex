\chapter{Evoluční algoritmy}
\section{Historie}
Začněme pohledem do historie Evolučních algoritmů na základě knih \citep{MitchellBook} a \citep{eibenIntro}. Darwinova myšlenka evoluce lákala vědce už před průlomem počítačů, Turing vyslovil myšlenku \textit{genetického a evolučního vyhledávání} už v roce 1948. V 50. a 60. letech nezávisle na sobě vznikají 4 hlavní teorie nesoucí podobnou myšlenku. Společným základem všech teorií byla evoluce populace kandidátů na řešení daného problému a jejich následná úprava způsoby hromadně nazývány jako genetické operátory, například mutace genů, přirozená selekce úspěšnějších řešení. \par 
Rechenberg a Schwefell (1965, 1973) představuje \textit{Evoluční strategie}, metoda optimalizující parametry v reálných číslech, jejich použití pro letadlová křídla. Fogel, Owens, Walsh zveřejňují \textit{evolutionary programming}(evoluční programování), technika využívající k reprezentaci kandidátů konečný automat(s konečným počtem stavů), který je vyvíjen mutací přechodů mezi stavy a následnou selekcí. \textit{Genetické algoritmy} vynalezl Holand v 60. letech a následně se svými studenty a kolegy z Michiganské Univerzity implementoval, oproti ES a EP nebylo hlavním cílem formovat algoritmus pro řešení konkrétních problémů, ale přenos obecného mechanismu evoluce jako metody aplikovatelné v informatickém světě. Princip GA spočívá v transformaci populace chromozonů(př. vektor 1 a 0) v novou populaci pomocí genetických operátorů křížení, mutací a inverze. V 1975 v knize \textit{Adaptation in Natural and  Artificial Systems} \citep{HolandBook} definoval genetický algoritmus jako abstrakci biologické evoluce spolu s teoretickým základem jejich používání. Ovšem někteří vědci používají pojem GA i ve významech hodně vzdálených původní Holandově definici. K sjednocení jednotlivých přístupů přispěl v 90. letehc Koza, dále jsou všechny zahrnuty jako oblasti \textit{Evolučních algoritmů}. Dnes existuje řada konferencí a odborných časopisů sdružující pracovníky zabývající se touto oblastí. Zmiňme ty větší z nich, co se týče konferencí: 
\href{http://gecco-2017.sigevo.org/index.html/HomePage}{GECCO}, \href{http://www.ppsn2016.org/conference}{PPSN}, 
\href{http://www.cec2017.org/}{CEC}, 
\href{http://www.evostar.org/2018/}{EVOSTAR}, 
časopisy: 
\href{http://www.mitpressjournals.org/loi/evco}{Evolutionary Computation}, 
\href{http://ieeexplore.ieee.org/xpl/RecentIssue.jsp?reload=true&punumber=4235}{IEEE Transactions on Evolutionary Computation}, 
\href{http://www.springer.com/computer/ai/journal/10710}{Genetic Programming and Evolvable Machines},
\href{https://www.journals.elsevier.com/swarm-and-evolutionary-computation/}{Swarm and Evolutionary Computation}
\section{Obecné principy}.