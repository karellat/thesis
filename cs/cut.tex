/section{Genetické programování} 
Genetické programování je kolekce metod v oboru umělé inteligence. V  GP vyvijíme počítočové populaci počítačových programů. Kde počítačový program představují geny(vektor čísel), hledaným výsledkem je nakonec program reprezentovaný konkrétním genem, který vhodně plní předem definovaný úkol. Prohledávání prostoru všech možných genů zajišťují evoluční algoritmy(též zvané genetické algoritmy), které populace genů modifikují, ohodnocují, dávají tím vzniknout nové lepší populaci. Tento proces se opakuje, dokud není splněna koncová podmínka. Metoda tedy zahrnuje vytvoření reprezentace programu jako vektoru čísel a dále hledání úspěšného genu v prostoru všech možných genů \citep{field}. Tyto metody se inspirují teorií evoluce(přirozený výběr), stejně jako v přírodě jedná se o náhodný proces, proto tedy není garantovaný výledek a GP řadíme k stochastickým metodám.

JOURNALs 
Evolutionary Computation,
IEEE Transactions on Evolutionary Computation, 
Genetic Programming and Evolvable Machines,
Evolutionary Intelligence, 
Swarm and Evolutionary Computing